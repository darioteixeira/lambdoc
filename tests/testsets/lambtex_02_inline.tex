Here we have examples of text in \bold{bold} and \emph{emphasised} and \code{monospaced}
and in \caps{small-caps} and with \sup{superscripts} and \sub{subscripts}. You may also
correct \del{perfect}\ins{broken} text.  Also, \mbox{\emph{all of this emphasised text
should appear on the same line.}} It is possible to enter \caps{HTML} entities either by
name (\code{\&euro;} produces '&euro;'), by decimal code point (\code{\&#8364;} produces
'&#8364;'), or by hexadecimal code point (\code{\&#x20ac;} produces '&#x20ac;').
You can also enter en-dashes (\code{\-\-} produces '--'), em-dashes (\code{\-\-\-}
produces '---'), and proper double quotes (ex: ``hello'').  The input charset is
\caps{utf-8}, so you can --- for example --- enter Japanese characters directly:
おなかがいっぱいですからなにもたべたくないです.

