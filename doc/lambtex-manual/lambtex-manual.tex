\newmacro{lambtex}{0}{\caps{lambtex}}
\newmacro{lambdoc}{0}{\caps{lambdoc}}

\newmacro{optional}{0}{\span<optional>{≈}}
\newmacro{mandatory0}{0}{\span<mandatory>{✔\sub{ø}}}
\newmacro{forbidden}{0}{\span<forbidden>{✘}}
\newmacro{forbidden0}{0}{\span<forbidden>{✘\sub{ø}}}
\newmacro{depends0}{0}{\mref{sec:order}{\forbidden0/\mandatory0}}

\newmacro{idem}{0}{\emph{(id.)}}


\title{The \lambtex Manual}
\subtitle{Dario Teixeira}
\subtitle{June 2016}

\toc[sec:toc]

\rule

\p<initial>{\lambtex is a document description language.  It is largely inspired
by \caps{LaTeX}\cite{bib:latex}, both in its syntax and in the approach taken for
the description of documents.  In basic terms, the source for a \lambtex document
is composed of plain \caps{utf-8} text annotated with special commands that attach a
precise semantical meaning to elements.  There are commands for formating passages in
a special manner (emphasised, superscript, etc), and commands that declare high-level
constructs such as section headings, quotation blocks, boxouts, or pictures.  \lambtex
was especially designed to be used in web applications, and as such it supports fully
hyperlinked documents.  Moreover, its syntax is simple, consistent, and designed with
brevity in mind.}

This manual is divided into two parts: a \mref{part:tutorial}{\lambtex tutorial} and
an \mref{part:theappendix}{appendix}.  The tutorial introduces all the principles
and features of the language, whereas the appendix provides reference information
concerning miscelaneous issues which do not fit into the main flow of the manual.


\part()[part:tutorial]{\lambtex tutorial}

\section[sec:anatomy]{The anatomy of \lambtex documents}

A \lambtex document is essentially a sequence of blocks.  A block is for example
a paragraph of text, a section heading, a figure, or a table.  Blocks typically
occupy the full width of the page and are vertically separated from other blocks by a
small margin.  In the document source, whitespace is irrelevant when it comes to the
separation of blocks.

Many blocks are composed of \emph{inline text}.  Paragraphs and section headings are good
examples of such blocks.  Inline text is a sequence of plain text and special characters
intermixed with sequences of emphasised text, superscript text, etc.  You can thus speak
of \emph{block-level} and \emph{inline-level} elements in a \lambtex document.

Inline-level elements are declared by \emph{simple} commands, while block-level elements
may be declared by either simple or \emph{environment} commands\see{note:commands}.
The former take the generic shape \mono{\\\emph{command}\{...\}}, where the content
inside the curly brackets constitutes the parameters to the command.  Note that some
simple commands may take more than one parameter (in which case more than one set of
curly brackets is used), while others exist that take no parameters at all.  As for
environment commands, they are composed of two parts: an initial and a final tag, which
together delineate the contents of the environment.  The initial tag takes the form
\mono{\\begin\{\emph{command}\}}, while the final tag is \mono{\\end\{\emph{command}\}}.

\begin[note:commands]{note}
As stated, any given command may either be expressed by the simple notation or as
an environment.  This should be interpreted an exclusive-\caps{or} condition, meaning
that no command exists that can be expressed as both.  However, the choice of adopting
one or the other notation was not an arbitrary one.  Due to its syntactic terseness,
the simple notation was preferred.  The more verbose environment notation was chosen
only for those commands that satisfied at least one of the two following conditions:

\begin{ul}
\li The command may be the parent of at least one block-level element.
\li The command declares a verbatim-like environment using the \caps{heredoc}
    technique for delimitation (see \sref{sec:verbatim} for an example).
\end{ul}
\end{note}

Besides the (normally mandatory) primary parameters, both simple and environment
commands accept three other kinds of (usually optional\see{note:optional}) parameters:
order parameters, label parameters, and style parameters.  The first are enclosed by
round brackets and are used to override the automatic ordering associated with blocks
such as sections and equations.  The second are enclosed by square brackets and are
used for assigning a label to any block that may be referenced from somewhere else
in the document.  Finally, style parameters are enclosed between angle brackets and
provide the mechanism for customising the appearance of individual document elements.
For the sake of reference, \sref{tab:parameters} lists all the classes of parameters
and the characters used for enclosing them.

\begin[note:optional]{note}
While the qualifier \emph{typically} implies a degree of uncertainty concerning the
optionality of these parameters, in practice you may treat them as truly optional.
The qualifier exists because there is one (and only one!) exception, which applies
to environments not normally used in regular documents.  Namely, inside sub-pages the
order parameter is mandatory for a number of blocks (see \sref{sec:order} for details).
\end{note}

\begin[tab:parameters]{table}
{The four classes of parameters that may be given to \lambtex commands.
Primary parameters are typically mandatory and are enclosed by curly brackets.
Order, label, and style parameters are usually optional and are enclosed by
round, square, and angle brackets, respectively.}
\begin<cols=ccc>{tabular}
\head
|           |<cell=2c_> Markers     |
| Parameter | Begin     | End       |
\body
| Order     | \mono{(}  | \mono{)}  |
| Label     | \mono{[}  | \mono{]}  |
| Style     | \mono{<}  | \mono{>}  |
| Primary   | \mono{\{} | \mono{\}} |
\end{tabular}
\end{table}

Style parameters come in two varieties: key-value pairs and classnames.  The former
variety takes the form '\mono{\emph{key}=\emph{value}}', and is available only for
some commands.  Moreover, what constitutes a valid key-value combination is command
dependent, and is described in more detail in the section that describes each of
those commands.  As for classname parameters, they map to \caps{html} classnames in
a straightforward way, and are available for every command.  If, for example, you
were to provide '\mono{important}' as the style parameter to a command, this would be
interpreted as a classname declaration, and translated into the \caps{html} classname
'\mono{doc_class_important}' (the prefix '\mono{doc_class_}' is preppended to every
custom classname declaration to prevent name clashes).  Note that classnames must
begin with a lowercase Roman letter, and may contain only lowercase Roman letters,
digits, the character '\mono{-}' (dash), or the character '\mono{_}' (underscore).

Bear in mind that \lambtex is agnostic on the validity and semantics of each
command/classname combination.  Nevertheless, \lambdoc --- the library that provides
\lambtex support for applications --- does have facilities for restricting the set
of acceptable command/classname combinations.  Therefore, you may find yourself using
\lambtex in an application that may forbid classname style parameters altogether, or that
at least severely limits their use.  Note also that because some block customisations
are so widespread, the default \caps{css} shipped with \lambdoc includes declarations
for the most common customisations.  These are described in detail in the section that
describes each command.

In the case of order and label parameters, the attributed value is necessarily singular,
as it makes no sense to give multiple orderings to the same command nor it is possible
to assign multiple labels to one command.  In the case of style parameters, however,
multiple attributions separated by a '\mono{,}' (comma) may be given.

Not all of the optional parameters are available for all commands.  There are in fact
many commands for which only style parameters are applicable.  Moreover, for those
commands where one or more optional arguments are accepted, they may be specified in
any order as long as they appear before any of the primary (curly bracketed) parameters.


\section[sec:paragraphs]{Paragraph blocks}

Paragraphs are the simplest of all blocks, typically being composed of just a sequence
of inline text terminated by a blank line.  It is however also possible to declare
a paragraph via the simple command \mono{\\paragraph} or its synonym \mono{\\p}.
Should you choose the simple command, the paragraph contents are to be specified inside
the mandatory curly brackets.  Consider thus the following \lambtex source fragment:

\begin<boxed>{source}
This is the first example of a paragraph.  Note that
it is composed of inline text.

Lines can be broken at any point, because a paragraph
is only terminated by a blank line. \br
It is however possible to manually insert a line break
without terminating the paragraph.

\p{There is also a simple command that allows the
explicit declaration of a paragraph.}
\end{source}

The result produced by the composer is as follows:

\begin{subpage}
This is the first example of a paragraph.  Note that
it is composed of inline text.

Lines can be broken at any point, because a paragraph
is only terminated by a blank line. \br
It is however possible to manually insert a line break
without terminating the paragraph.

\p{There is also a simple command that allows the
explicit declaration of a paragraph.}
\end{subpage}

Note that the line breaks within the text are irrelevant for the composer; all that
it cares about are that blank lines separate paragraphs.  if you wish for a line break
within a paragraph (ie, without terminating the current paragraph), use the \mono{\\br}
simple command, as demonstrated.

Using the command \mono{\\paragraph} (or its synonym \mono{\\p}) to declare a
paragraph is usually neither necessary nor recommended.  This method exists for the
sole reason that it may be desirable to provide the composer with style parameters
for the paragraph. 

The default \caps{css} shipped with \lambdoc supports three style customisations
for paragraphs.  The associated classnames are \mono{initial}, \mono{indent},
and \mono{noindent}.  The first declares that the first letter of the paragraph
should be rendered as an \link{http://en.wikipedia.org/wiki/Initial}{initial}.
The second and third indicate whether the default paragraph indentation rules should
be overriden.  By default, a paragraph is only indented when immediately preceded by
another paragraph.  By providing the \mono{indent} classname as a style parameter,
you are declaring that the paragraph should always be indented regardless of context.
Similarly, the \mono{noindent} classname indicates that the paragraph should not be
indented no matter what.


\section[sec:inline]{Formatting inline text}

As shown by \sref{tab:inline}, \lambtex offers a broad set of text formatting commands.
Remember that besides their application inside paragraph blocks, any of these commands
may also be used in a context where inline text is allowed.  That means you may use
them in section titles, figure captions, inside tables, etc.  Note that with the
exception of \mono{\\code} --- which takes raw text as parameter --- all inline formatting
commands take as parameter a sequence of inline text.  It is thus perfectly possible to
nest these commands without any restrictions.  Moreover, some of these commands have
(typically shorter) synonyms.  Therefore, instead of \mono{\\bold} you can also just
type \mono{\\b}.

\begin[tab:inline]{table}
{List of all text formatting commands, including synonyms where
applicable.  All these commands take one parameter enclosed in
curly brackets.  The parameter is a sequence of inline text.}
\begin<cols=rl>{tabular}
\head
| Command                                   | Example                                                       |
\body
| \mono{\\bold\{\emph{inline}\}}            | You can make some passages be \bold{bold}.\br
                                              (Synonyms: \mono{\\strong} and \mono{\\b})                    |
| \mono{\\emph\{\emph{inline}\}}            | Here's another way to \emph{emphasise} text.\br
                                              (Synonyms: \mono{\\em} and \mono{\\i})                        |
| \mono{\\mono\{\emph{inline}\}}            | You can also imitate a \mono{typewriter}.\br
                                              (Synonym: \mono{\\tt})                                        |
| \mono{\\caps\{\emph{inline}\}}            | \lambtex supports text in small-caps.                         |
| \mono{\\ins\{\emph{inline}\}}             | Sometimes it's \ins{very} handy to insert new text.           |
| \mono{\\del\{\emph{inline}\}}             | You may also fix \del{annoying} incorrect passages.           |
| \mono{\\sup\{\emph{inline}\}}             | The 1\sup{st} of January has a superscript.                   |
| \mono{\\sub\{\emph{inline}\}}             | H\sub{2}O requires a subscript.                               |
| \mono{\\mbox\{\emph{inline}\}}            | \mbox{This inline sequence should not be split across lines.} |
| \mono{\\span\{\emph{inline}\}}            | \span{You may create spans with custom semantics.}            |
| \mono{\\code\{\emph{raw}\}}               | You can use syntax-highlighted code like 
                                              \code<lang=caml>{type t = Alpha | Beta} in an inline context.
                                              See \sref{sec:code} for details.                              |
\end{tabular}
\end{table}

While most commands listed in the table above should have a straightforward application,
a few clarifying words concerning \mono{\\span} are in order.  This command translates
directly into an \caps{xhtml} \mono{span} element, and exists for the sole purpose
of associating a span of text with a custom semantic meaning beyond those listed in
\sref{tab:inline}.


\section{Special characters}

\lambtex allows the direct use of \caps{html} entities via the same notation used
in \caps{html}.  More precisely, entities are to be enclosed between the characters
'\mono{\&}' (ampersand) and '\mono{;}' (semicolon), and may be specified by the so-called
\emph{character entity reference} (ie, the entity's name), or by \emph{numeric character
reference} (ie, a \caps{unicode} code point in decimal or hexadecimal notation).
As an example, if the \caps{euro} symbol '\mono{&euro;}' is not readily available in
your keyboard, you may enter it in either of the following three manners:

\begin<cols=ccc>{tabular}
\head
| Name              | Decimal           | Hexadecimal       |
\body
| \mono{\&euro;}    | \mono{\&#8364;}   | \mono{\&#x20ac;}  |
\end{tabular}

The reader will have noted that \lambtex must interpret the character '\mono{\&}'
(ampersand) differently, and therefore you need a workaround if you want to input
an actual ampersand.  In fact, there are other characters besides the ampersand that
are interpreted differently by \lambtex, and as such also require a special escaping
sequence.  These characters are '\mono{\\}' (backslash), '\mono{\{}' (opening curly
bracket), '\mono{\}}' (closing curly bracket), and '\mono{\|}' (pipe).  The character
used for escaping is the backslash, which means that displaying an ampersand, for
example, is achieved with \mono{\\\&}.  Another option in this particular case is of
course to use the \mono{\&amp;} \caps{html} entity\see{note:ampersand}.

\begin[note:ampersand]{note}
Note that in most practical situations you can get away with inputting an ampersand
directly without escaping.  This is because the scanner will only interpret an ampersand
as the beginning of an entity declaration if it is immediately followed by an alphabetic
or numeric sequence terminated by a semicolon.  If it is not, then the ampersand is
interpreted as plain text.
\end{note}

Besides individual characters, there are also some multi-character sequences that are
interpreted differently by \lambtex.  \sref{tab:seq} lists all of them.  Remember to
use the escape character if you wish to input any of these sequences literally.

\begin[tab:seq]{table}
{Special character sequences and their \lambtex interpretation.  Remember
to use the backslash for escaping if you wish to input any of these sequences
verbatim.  Also, see \sref{sec:math} for details about entering mathematics.}
\begin<cols=cl>{tabular}
\head
| Sequence      | Translation |
\body
| \mono{\$\$}   | Delimiter for inline \caps{TeX} mathematics |
| \mono{\-\-\-} | Typographic em-dash |
| \mono{\-\-}   | Typographic en-dash |
| \mono{\`\`}   | Opening double quotes |
| \mono{\'\'}   | Closing double quotes |
\end{tabular}
\end{table}

What if the glyph you wish to enter is not part of the \caps{unicode} character set?
The solution in this case is to declare an inline image; this feature is covered in
detail in \sref{sec:glyphs}.


\section{Hyperlinks to external resources}

The simple command \mono{\\link} (or its synonym \mono{\\a}) allows
the creation of hyperlinks into any external resource.  It takes the form
\mono{\\link\{\emph{uri}\}\{\emph{inline}\}}, where \mono{\emph{uri}} is the raw text
containing the address (\caps{uri}) of the target, and \mono{\emph{inline}} is the
sequence of inline text that will actually be displayed.  Should the two be the same,
you may ommit the second parameter altogether.  Moreover, note that in no circumstances
may you nest a link inside another!\see{note:links}

\begin[note:links]{note}
The \caps{xhtml} standard forbids nested links.  Curiously enough, however, the \caps{w3c}'s
own famous \link{http://validator.w3.org/}{validator} is unable to detect such violations
of the specification.
\end{note}

Note that in the preceeding paragraph, the sequence \mono{\emph{uri}} is described as
\emph{raw} text, not an inline sequence.  In brief, a raw sequence is one whose contents
are parsed verbatim, without (almost) any attempt to expand special characters and where
inline formatting commands do not apply.  In similarity to their inline counterparts,
raw sequences are delimited by curly brackets.  Therefore, the end of sequence character
'\mono{\}}' is interpreted specially.  Escaping is also possible, and thus the backslash
character is also special.

\lambtex supports also internal targets and hyperlinks.  These are discussed in \sref{sec:intlinks}.

\section{List environments}

All the commands we have seen so far are simple commands, and therefore the declaration
of lists will provide us with the first example of environment commands.  Lists in
\lambtex come in three varieties: ordered, unordered, and description lists.
These are declared using the environment commands \mono{enumerate}, \mono{itemize},
and \mono{description}, respectively.  As shown by the following \lambtex source,
items in ordered and unordered lists must be preceded by the parameterless simple
command \mono{\\item}.  As for description lists, each item must be preceded also by
the simple command \mono{\\item}.  However, in this latter case the command takes a
mandatory inline sequence as parameter.  Finally, lists may be nested, and each item may
itself be composed of several blocks (see \sref{sec:structure} for a detailed description
concerning the structure of \lambtex documents).  Consider thus the following source:

\begin<boxed>{source}
Here is a list of small furry mammals:

\begin{itemize}
\item Cat
\item Some dog breeds:
    \begin{description}
    \item{poodle:} Small dogs kept for company
    \item{greyhound:} Running dogs
    \end{description}
\item Various rodents ordered by size:
    \begin{enumerate}
    \item Capivara
    \item Rat
    \item Mouse
    \end{enumerate}
\end{itemize}
\end{source}

The output generated by the composer is as follows:

\begin{subpage}
Here is a list of small furry mammals:

\begin{itemize}
\item Cat
\item Some dog breeds:
    \begin{description}
    \item{poodle:} Small dogs kept for company
    \item{greyhound:} Running dogs
    \end{description}
\item Various rodents ordered by size:
    \begin{enumerate}
    \item Capivara
    \item Rat
    \item Mouse
    \end{enumerate}
\end{itemize}
\end{subpage}

For the sake of brevity and for compatibility with \caps{html} habits, you may also use
the synonyms \mono{ol}, \mono{ul}, \mono{dl}, and \mono{\\li}, for \mono{enumerate},
\mono{itemize}, \mono{description}, and \mono{\\item}, respectively.  Moreover,
the British spelling \mono{itemise} may be used for ordered lists.  However, the
start and end tags must still match: though, for example, \mono{ol} is a synonym for
\mono{enumerate}, if you start an ordered list with \mono{\\begin\{ol\}} you must
terminate it with \mono{\\end\{ol\}}.

The default \caps{css} shipped with \lambtex allows for the customisation of
the bullet used in unordered lists or the numbering system used in ordered lists.
To do so you need to pass a style parameter to either the \mono{\\begin\{itemize\}}
or \mono{\\begin\{enumerate\}} commands.  Below is the list of classnames available
out-of-the-box for itemize environments:

\begin<cols=rl>{tabular}
\head
| Value         | Type of bullet                |
\body
| \mono{disc}   | A filled circle.              |
| \mono{circle} | An empty circle.              |
| \mono{square} | A square.                     |
| \mono{none}   | Ommit the bullet altogether.  |
\end{tabular}

And the corresponding list for itemize environments:

\begin<cols=rl>{tabular}
\head
| Value                 | Type of numbering                         |
\body
| \mono{decimal}        | Decimal numbers.                          |
| \mono{lower-roman}    | Lower-case Roman numerals.                |
| \mono{upper-roman}    | Upper-case Roman numerals.                |
| \mono{lower-alpha}    | Lower-case letters in the Latin alphabet. |
| \mono{upper-alpha}    | Upper-case letters in the Latin alphabet. |
| \mono{none}           | Ommit the numbering altogether.           |
\end{tabular}


\section[sec:qanda]{Q&A environments}

Somewhat related to lists, Q&A environments are used to mark a sequence of questions and
accompanying answers, as typically found in interviews.  The environment is bounded by
the start/end tag \mono{qanda}, and expects a sequence of question and answer blocks,
delimited by the simple commands \mono{\\question} and \mono{\\answer}, respectively.
Both these commands accept an inline sequence delimited by curly brackets as optional
parameter.  If present, this sequence indicates the text that should appear before
the actual question or answer; it is used to indicate the name of the interviewer or
the interviewee, respectively.  Should this sequence be fixed for the entire length
of the Q&A, there is an alternative to repeating it with each \mono{\\question}
and\mono{\\answer} pair: you may instead use the parameterless simple commands
\mono{\\rquestion} and \mono{\\ranswer}.  These simply reuse the sequence declared in
the previous question or answer.  The following example illustrates the pattern:

\begin<boxed>{source}
\begin{qanda}
\question{Egg Gazette:}
On which side of the egg opening debate are you?

\answer{John McAuthor:}
I sustain that eggs should be opened by their small-end, of course.

\question{EG:}
What's your opinion on the contrarian position,
that eggs should be opened by their large-end?

\answer{JM:}
Sheer madness!

\rquestion
Could you elaborate?

\ranswer
I don't believe that to be necessary; the reasons are self-evident.
\end{qanda}
\end{source}

The generated output is as follows:

\begin{subpage}
\begin{qanda}
\question{Egg Gazette:}
On which side of the egg opening debate are you?

\answer{John McAuthor:}
I sustain that eggs should be opened by their small-end, of course.

\question{EG:}
What's your opinion on the contrarian position,
that eggs should be opened by their large-end?

\answer{JM:}
Sheer madness!

\rquestion
Could you elaborate?

\ranswer
I don't believe that to be necessary; the reasons are self-evident.
\end{qanda}
\end{subpage}


\section[sec:verse]{Verse environments}

The environment \mono{verse} is used to typeset poetry, music lyrics, or any other
content where conventional typographic rules for verses apply.  Inside the environment
only paragraph blocks are accepted; each paragraph corresponds to a stanza, and each
verse should be terminated by the line break marker \mono{\\br}.  Therefore, if you
wished to typeset the first couple of stanzas from Virgil's \emph{Aeneid}, this would
be the \lambtex source:

\begin<boxed>{source}
\begin{verse}
Arma virumque cano, Troiae qui primus ab oris \br
Italiam, fato profugus, Laviniaque venit \br
litora, multum ille et terris iactatus et alto \br
vi superum saevae memorem Iunonis ob iram; \br
multa quoque et bello passus, dum conderet urbem, \br
inferretque deos Latio, genus unde Latinum, \br
Albanique patres, atque altae moenia Romae.

Musa, mihi causas memora, quo numine laeso, \br
quidve dolens, regina deum tot volvere casus \br
insignem pietate virum, tot adire labores \br
impulerit.  Tantaene animis caelestibus irae? \br
\end{verse}
\end{source}

The result would be the following:

\begin{subpage}
\begin{verse}
Arma virumque cano, Troiae qui primus ab oris \br
Italiam, fato profugus, Laviniaque venit \br
litora, multum ille et terris iactatus et alto \br
vi superum saevae memorem Iunonis ob iram; \br
multa quoque et bello passus, dum conderet urbem, \br
inferretque deos Latio, genus unde Latinum, \br
Albanique patres, atque altae moenia Romae.

Musa, mihi causas memora, quo numine laeso, \br
quidve dolens, regina deum tot volvere casus \br
insignem pietate virum, tot adire labores \br
impulerit.  Tantaene animis caelestibus irae? \br
\end{verse}
\end{subpage}


\section[sec:quotations]{Quotations}

Quotation blocks are typically used for including an excerpt from an external source
into your document.  You may be quoting what someone said in an interview, or providing
context when replying to a comment, for example.  Quotation bloks are delimited by
the \mono{quote} environment, and may enclose not only plain text paragraphs, but
also lists or any of the so-called \emph{quotable blocks}, as \sref{sec:structure}
describes in greater detail.  Consider thus the following \lambtex source:

\begin<boxed>{source}
This is what you said:

\begin{quote}
Eggs should be opened by their large end.  Any other method is heresy.
It is therefore my opinion that John McAuthor is a heretic.
\end{quote}
\end{source}

This is the output produced:

\begin{subpage}
This is what you said:

\begin{quote}
Eggs should be opened by their large end.  Any other method is heresy.
It is therefore my opinion that John McAuthor is a heretic.
\end{quote}
\end{subpage}

Note that though undoubtedly indispensable, quotation blocks are prone to being abused.
Our advice is to use them only if embeddeding a quotation is essential to carry your
message.  You may, for example, be commenting on a particular sentence or paragraph
in an interview available online; in this case, a quotation block will provide the
focus that would have been lost by simply linking to the entire interview.


\section[sec:verbatim]{Verbatim environments}

Like their \caps{LaTeX} counterparts or the \caps{html} \mono{pre} element, verbatim
environments allow the insertion of blocks where the text is shown \emph{as is}, ie,
without (almost) any attempt at the interpretation of its contents.  They are to be
displayed using a monospaced font, and as such are ideal for showing \caps{ascii}-art.
Another application is to display text-only technical diagrams as commonly found on
Internet \caps{rfc}s, for example.

The command name for the verbatim environment is \mono{verbatim}, and being an
environment command the text must be enclosed between \mono{\\begin\{verbatim\}} and
\mono{\\end\{verbatim\}} markers.  The synonym \mono{pre} may also be used.  Below is
an example of a diagram drawn using only \caps{ascii} characters, and rendered using
a \mono{verbatim} environment:

\begin{verbatim}
             N
            /|\
           / | \
          /  |  \
         /   |   \
        /    |    \
       /     |     \
      N     '+'     L
     /|\            |
    / | \           |
   /  |  \          |
  /   |   \         |
 L   '+'   L       '3'
 |         |
'1'       '2'
\end{verbatim}

We say there is \emph{almost} no interpretation of the contents of the verbatim
environment because there is one exception.  The string \mono{\\end\{verbatim\}}
is used as the end marker, and therefore the \lambtex scanner will exit the
environment if it encounters this string.  This raises the obvious question of how
one can insert that string itself within a verbatim context without triggering the
termination of the environment.  The solution used by \lambtex is reminescent of the
\link{http://en.wikipedia.org/wiki/Here_document}{\caps{heredoc}} approach used by
other languages.  Basically, the markers that delimit verbatim environments need only use
\mono{verbatim} as a prefix.  The only condition for termination is that the end marker
matches the begin marker.  Therefore, to include the string \mono{\\end\{verbatim\}}
within a verbatim environment, you can simply delimitate that environment with, for
example, \mono{\\begin\{verbatim1\}} and \mono{\\end\{verbatim1\}}.  Note that only
lowercase Roman letters, digits, and the character '\mono{_}' (underscore) may be used
as suffix characters.

The \caps{css} shipped with \lambtex defines a series of classnames for exclusive
use with verbatim environments.  These are \mono{mult0}, \mono{mult1}, and so forth
until \mono{mult9}.  The integer indicates a multiplying factor for the contents of
the environment.  For a factor of $$x$$, the actual scaling relative to normal size is
given by $$\sqrt{2}^x$$.  The default factor is 0, which corresponds to normal size.
One common application of this feature is to display \caps{unicode} characters as
full-sized pictures.  As an example, this is what the \caps{unicode} snowman (code
point \caps{U+2603}) looks like with a multiplier of 7, corresponding to a font size
11.3 times larger than normal:

\begin<mult7>{verbatim}
☃
\end{verbatim}

In addition, the default \caps{css} also allows floatation specifiers to be used with
verbatim blocks.  Floating specifiers take the form of classname style parameters
'\mono{center}', '\mono{left}', and '\mono{right}'.  They declare that the block should
be displayed either centered and occupying the entire width of the page (this is the
default), floating on the left, or floating on the right, respectively.  Note that
they map directly to \caps{css} \mono{float} declarations.

Floatation specifiers are not exclusive to verbatim environments, and in subsequent
sections we shall encounter other block types that also accept them.  The property
all these block types have in common is the reasonable expectation that they may be
placed outside the main flow of the document without a loss of context.  In a sense,
for a block to be floatable, it must be able to carry its own context.  This is the
case for verbatim blocks, since they can be used for purely decorative purposes.


\section[sec:code]{Presenting source-code}

\sref{sec:inline} briefly introduced the simple inline command \mono{\\code}, whose
purpose is to format with syntax-highlighting a short sequence of source-code.
The only configuration available for this command is the optional key-value style
parameter '\mono{lang}', which allows the specification of the language used for
syntax-highlighting.

Suppose now that you have a long source-code listing that you wish to pretty-print.
While it is of course possible to input source-code listings using verbatim environment
discussed above, \lambtex features a dedicated facility that besides being semantically
more appropriate has in addition a number of tricks up its sleave.  This new environment
is named \mono{source}, and offers the possibility of adding line numbers to the code
listing and highlighting the source according to the syntax of most popular languages.
You have in fact already encountered this environment multiple times in this document: it
has been used for formatting \lambtex source code, albeit without line numbers or syntax
highlighting.  Suppose thus you wish to pretty-print an \link{http://ocaml.org/}{OCaml}
fragment.  You can declare a source-code environment the following way:

\begin<boxed>{source1}
\begin<lang=caml>{source}
type 'a tree =
    | Leaf
    | Node of 'a tree * 'a * 'a tree

let rec count = function
    | Leaf                  -> 0
    | Node (left, x, right) -> 1 + count left + count right
\end{source}
\end{source1}

Note that the source language is specified via an optional key-value style parameter
'\mono{lang}'.  Also, the same \caps{heredoc}-like technique used for verbatim
environments can be used in a source environment if you wish to include the string
\mono{\\end\{source\}} itself.  The astute reader will have noticed that this trick
must have been used in the source for this manual to display the code above (this
manual is itself written in \lambtex --- we do eat our own dog food).  As for the
actual result produced by the composer, it is as follows:

\begin<lang=caml>{source}
type 'a tree =
    | Leaf
    | Node of 'a tree * 'a * 'a tree

let rec count = function
    | Leaf                  -> 0
    | Node (left, x, right) -> 1 + count left + count right
\end{source}

Another optional key-value style parameter accepted by source environments is '\mono{nums}'.
It accepts a boolean value, and dictates whether line numbers should be added to the
source code listing.  Valid boolean values are '\mono{true}', '\mono{yes}' and '\mono{on}'
for truth value \emph{true} and '\mono{false}', '\mono{no}' and '\mono{off}' for truth
value \emph{false}.  It defaults to false.

The \caps{css} shipped with \lambtex offers additional customisation options for the
appearance of source environments.  These are all available as classname style parameters,
as follows:

\begin{description}

\item{'\mono{plain}'}
This classname indicates that the listing should be displayed in plain style.  This is
the default, as previously shown.  It is better suited for source-code fragments that
one wishes to integrate tightly with the flow of the document.

\item{'\mono{boxed}'}
Declares that the source code listing should be shown inside a decorative box.  Example:

\begin<lang=caml,boxed>{source}
type 'a tree =
    | Leaf of 'a
    | Node of 'a tree * 'a tree

let rec count_leaves = function
    | Leaf _      -> 1
    | Node (a, b) -> count_leaves a + count_leaves b
\end{source}

\item{'\mono{zebra}'}
Besides placing the source-code inside a decorative box, this style will in addition
give a different background to even and odd numbered rows:

\begin<lang=caml,zebra>{source}
type 'a tree =
    | Leaf of 'a
    | Node of 'a tree * 'a tree

let rec count_leaves = function
    | Leaf _      -> 1
    | Node (a, b) -> count_leaves a + count_leaves b
\end{source}

\item{'\mono{console}'}
This style is appropriate if the code block should actually mimick the output from a
command line console:

\begin<console>{source}
dario@localhost:~/lambtex$ make
lambcmd -f lambtex -t html -i sample.lambtex -o index.html

dario@localhost:~/lambtex$ file index.html
index.html: HTML document text
\end{source}
\end{description}

Finally, note that \lambtex offers also the possibility for code environments to be
numbered and referenced from anywhere in the document.  This can be done by wrapping
them inside a \mono{printout} environment, and is part of the advanced features of
the language discussed in \sref{sec:wrappers}.


\section{Tabular environments}

Presenting data in a tabular format is indispensable for many classes of documents.
With this in mind, \lambtex offers a special \mono{tabular} environment which
allows you to format tables in a simple and concise way, while retaining most of the
capabilities offered by \caps{LaTeX} and \caps{xhtml} tables.

Suppose then that our tabular data had three different attributes --- a person's
name, a phone number, and an address --- and that we had this information about four
different people.  We wish to present the attributes as table columns, whereas each
record will be shown in a different row.  In brief, a table just like this one:

\begin{tabular}
\head
| Name      | Phone     | Address   |
\body
| John      | 123456    | London    |
| Mary      | 234567    | Paris     |
| Mark      | 345678    | Moscow    |
| Jane      | 456789    | Sydney    |
\end{tabular}

Inside a tabular environment, a row is terminated when the character '\mono{\|}' (pipe)
is the last on a line of text; within each row, each cell must also be prefixed by the
'\mono{\|}' character.  This means that you'll need to escape it if you wish to display
an actual pipe character.

There are three different classes of rows available: header, body, and footer rows.
Besides formatting each one differently, smart enough browsers will know that table
headers are to be repeated if the tabular environment spans more than one printed page.
The parameterless simple commands \mono{\\head}, \mono{\\body}, and \mono{\\foot}
are used inside a tabular environment to mark the beginning of each class.  Note that
header and footer rows are optional (in which case the \mono{\\body} marker is also
optional, since rows are assumed by default to be body rows).  Moreover, you can
declare more than one group of body rows, and the default \caps{css} stylesheet will
visibly separate each group.

The code below is the source for the tabular environment listed above.  Note the use
of the \mono{\\head} and \mono{\\body} commands to delineate row classes, and how the
character '\mono{\|}' is used to separate cells and rows.  Note also that all rows
have the same number of columns.  This is \bold{mandatory}, though provisions exist
for making a cell span multiple columns (see below).

\begin<boxed>{source}
\begin{tabular}
\head
| Name      | Phone     | Address   |
\body
| John      | 123456    | London    |
| Mary      | 234567    | Paris     |
| Mark      | 345678    | Moscow    |
| Jane      | 456789    | Sydney    |
\end{tabular}
\end{source}

All the columns in the tabular environment above are left-aligned and the column text
is shown in normal text.  You may, however, specify otherwise using the key-value
style parameter '\mono{cols}'.  Its value consists of a string whose length equals
the number of columns and where each character specifies the formatting that should
be applied to each column.  The example above features three left-aligned columns;
the corresponding style parameter is '\mono{cols=lll}'.  If, for example, you had four
columns aligned respectively to the left, right, center, and left again, the style
parameter would be '\mono{cols=lrcl}'. Consult \sref{tab:tabular} for a list of all
available column specifiers.

\begin[tab:tabular]{table}
{List of available column specifiers for tabular environments.
A specifier tells \lambtex how each column should be formatted.}
\begin<cols=rl>{tabular}
\head
| Specifier | Formatting                            |
\body
| \mono{l}  | Column is left-aligned.               |
| \mono{L}  | Column is left-aligned and strong.    |
| \mono{r}  | Column is right-aligned.              |
| \mono{R}  | Column is right-aligned and strong.   |
| \mono{c}  | Column is centre-aligned.             |
| \mono{C}  | Column is centre-aligned and strong.  |
| \mono{j}  | Column is justified.                  |
| \mono{J}  | Column is justified and strong.       |
\end{tabular}
\end{table}

Some tables may require more sophisticated formatting.  The most common requirement is
for multi-column cells, and \lambtex does support them.  This feature is enabled by a
special syntax that allows the customisation of each individual cell.  More precisely,
the pipe character '\mono{\|}' that precedes each cell is optionally followed by an
angle bracketed style parameter.

Recall that style parameters come in two varieties: classnames and key-value pairs.
You may of course use the former to customise the \caps{css} of an individual cell.
As for the latter, cells offer a single option, whose key is named '\mono{cell}', and
whose value is a formatting expression composed of three parts.  The first part is
an integer indicating how many columns the cell should span; the second is a single
character matching one of the specifiers listed in \sref{tab:tabular}; last and
optional are the characters '\mono{_}' (underscore) and/or '\mono{^}' (caret), which
if present indicate that a horizontal line should be put above and/or below the cell.
Note that the formatting should satisfy the regular expression \mbox{\mono{\emph{digit}+
\emph{spec} ( '_' \| '^' \| '_^' \| '^_' )?}}.

This discussion would not be complete without the presentation of a complete example
illustrating the application of advanced cell formatting.  Consider thus the \lambtex
fragment below, whose output is displayed in \sref{tab:species}.  Besides the use of
header, footer, and multiple body rows, also noteworthy are the cell style parameters
\mono{cell=2c_} and \mono{cell=2c^}, which configure their corresponding cells to span
two columns, be centered, and show a horizontal line below and above, respectively.

\begin<boxed>{source}
\begin<cols=rll>{tabular}
\head
|               |<cell=2c_> Scientific name              |
| Common name   | Genus                 | Species        |
\body
| Wolf          | \emph{Canis}          | \emph{lupus}   |
| Cat           | \emph{Felis}          | \emph{catus}   |
| Lion          | \emph{Panthera}       | \emph{leo}     |
\body
| English oak   | \emph{Quercus}        | \emph{robur}   |
| Cork oak      | \emph{Quercus}        | \emph{suber}   |
| European yew  | \emph{Taxus}          | \emph{baccata} |
\foot
| Common name   | Genus                 | Species        |
|               |<cell=2c^> Scientific name              |
\end{tabular}
\end{source}

To conclude, note that if you wish for a tabular environment to be numbered and
referenced from anywhere in the document, you need to wrap it inside a \mono{table}.
This is part of the advanced features of the language, and is discussed in
\sref{sec:wrappers}.

\begin[tab:species]{table}
{The classification of some common animals and plants.  This table illustrates the use
of the advanced formatting facilities that enable customisation of individual cells.}
\begin<cols=rll>{tabular}
\head
|               |<cell=2c_> Scientific name              |
| Common name   | Genus                 | Species        |
\body
| Wolf          | \emph{Canis}          | \emph{lupus}   |
| Cat           | \emph{Felis}          | \emph{catus}   |
| Lion          | \emph{Panthera}       | \emph{leo}     |
\body
| English oak   | \emph{Quercus}        | \emph{robur}   |
| Cork oak      | \emph{Quercus}        | \emph{suber}   |
| European yew  | \emph{Taxus}          | \emph{baccata} |
\foot
| Common name   | Genus                 | Species        |
|               |<cell=2c^> Scientific name              |
\end{tabular}
\end{table}


\section[sec:math]{Mathematical formulas}

\lambtex also supports the inclusion of equations and mathematical formulas in a
document.  Mathematics are to be entered using either \caps{MathML}\cite{bib:mathml}
or the \caps{TeX}\cite{bib:tex} format.  The former was chosen because an increasing
number of tools has the ability to export \caps{MathML}.  As for the latter, it is
the \emph{de facto} standard for manually typing mathematical expressions.

Mathematical expressions may be entered in an inline context (embedded inside a
paragraph, for example), or in a block context.  In the former case, the simple commands
\mono{\\mathtexinl} and \mono{\\mathmlinl} may be used for inputting mathematics in
the \caps{TeX} or \caps{MathML} formats, respectively.  For the sake of convenience,
\caps{TeX} mathematics may also be entered by delimiting them between the characters
'\mono{\$\$}'.  Though \lambtex does not enforce it, common sense dictates that
inline mathematics should restrict themselves to expressions that fit comfortably in
a single line.  One example is the quadratic equation $$ax^2 + bx + c = 0$$.

\caps{TeX} and \caps{MathML} mathematical expressions in a block context are specified
via the \mono{mathtexblk} and \mono{mathmlblk} environments, respectively.  Furthermore,
because both these environments expect their content to be raw text that should be parsed
verbatim, they accept the \caps{heredoc} technique for delimiting the environment.
Suppose thus you wished to enter a math block containing the quadratic formula that
solves the equation listed above.  Specified in the \caps{TeX} format, the code would
be the following:

\begin<boxed>{source}
\begin{mathtexblk}
x = \frac{-b \pm \sqrt{b^2 - 4ac}}{2a}
\end{mathtexblk}
\end{source}

The equivalent Presentation \caps{MathML} markup is listed below.  Note that it is
also acceptable to provide Content \caps{MathML} instead of Presentation \caps{MathML}.
In either case, just make sure the markup conforms to version 2.0 of the \caps{MathML}
standard, and that the \mono{math} top-level element that wraps the \caps{MathML}
contents is provided as part of the markup.  These requirements also apply to the
\caps{MathML} markup specified in an inline context.  On the other hand, giving
an explicit namespace to the \caps{MathML} markup is \bold{not} necessary, since
namespacing requirements are automatically taken care of.

\begin<boxed>{source}
\begin{mathmlblk}
<math>
    <mrow>
        <mi>x</mi>
        <mo>=</mo>
        <mfrac>
        <mrow>
            <mo form="prefix">&minus;</mo>
            <mi>b</mi>
            <mo>&PlusMinus;</mo>
            <msqrt>
            <msup>
                <mi>b</mi>
                <mn>2</mn>
            </msup>
            <mo>&minus;</mo>
            <mn>4</mn>
            <mo>&InvisibleTimes;</mo>
            <mi>a</mi>
            <mo>&InvisibleTimes;</mo>
            <mi>c</mi>
            </msqrt>
        </mrow>
        <mrow>
            <mn>2</mn>
            <mo>&InvisibleTimes;</mo>
            <mi>a</mi>
        </mrow>
        </mfrac>
    </mrow>
</math>
\end{mathmlblk}
\end{source}

The equation resulting from either of the above sources is shown below.  To conclude,
bear in mind that in similarity with other types of blocks, \lambtex also supports
named mathematical expressions that may be numbered and referenced from anywhere
in a document.  For that you need to wrap a \mono{mathtexblk} or \mono{mathmlblk}
environment inside an \mono{equation} block.  This technique is part of the advanced
features of the language and is discussed with greater detail in \sref{sec:wrappers}.

\begin{mathtexblk}
x = \frac{-b \pm \sqrt{b^2 - 4ac}}{2a}
\end{mathtexblk}


\section{Images}

\lambtex allows the insertion of images either in inline contexts or in block contexts.
In the former case, the image is typically the representation of a character which is
unavailable in the \caps{unicode} set, and is for this reason denoted as a \emph{glyph}.
The latter case includes images used as illustrations, diagrams, photographs, etc.
These are known as \emph{pictures}.


\subsection[sec:glyphs]{Glyphs}

The simple command \mono{\\glyph} is used for inserting an image into an inline context.
It takes the form \mono{\\glyph\{\emph{identifier}\}\{\emph{alt}\}\{\emph{title}\}},
where \mono{\emph{identifier}} is the name of the glyph (typically the filename), and
\mono{\emph{alt}} is the alternative text to be displayed if the browser cannot or is not
configured to display images.  Both parameters consist of raw text and are mandatory.
The final parameter \mono{\emph{title}} consists also of raw text, but is optional.
It can be used to provide a small description which pops up when the user hovers the
mouse over the image.

Because glyphs are embedded into a paragraph or other inline context, the image should
be of comparable size to the text.  This is simply a recommendation, however, as it
is not enforced.


\subsection{Pictures}

The insertion of block images into a document is done via the simple
command \mono{\\picture}, which in its basic usage is identical
to the afore described \mono{\\glyph}.  Namely, it takes the form
\mono{\\picture\{\emph{identifier}\}\{\emph{alt}\}\{\emph{title}\}}, where
\mono{\emph{identifier}} is the name of the picture (typically the filename), and
\mono{\emph{alt}} is the alternative text to be displayed if the browser cannot or is
not configured to display images.  Both parameters consist of raw text and are mandatory.
The final parameter \mono{\emph{title}} consists also of raw text, but is optional.

The command \mono{\\picture} differs from \mono{\\glyph} in that it must be used
in a block context and that it supports some additional configuration via key-value
style parameters.  The bitmap below illustrates the use of a image block.  Please do
try hovering the mouse over the image!

\picture{circle.png}{This is a circle}{Thank you for hovering!}

Note that in most browsers you cannot see the alternative text unless the browser is
configured not to display images.  As for the code that produced the picture, it is as
simple as this:

\begin<boxed>{source}
\picture{circle.png}{This is a circle}{Thank you for hovering!}
\end{source}

The size of the picture can be controlled by the optional key-value style parameter
\mono{width}.  Its value is either \mono{auto} or an integer between 1 and 100
(inclusive), indicating the percentage of the total width of the page that the picture
should occupy (minus a small margin common to all non-text blocks).  The default is
\mono{auto}, which translates into a picture occupying its natural width in pixels
(up to the maximum width of the page).  Note that block images are always clickable and
that the target \caps{uri} is that of the picture itself.  This is useful, for example,
for seeing at their full size pictures that were scaled down with the \mono{width}
parameter or that did not fit into the width of the page.

The default \caps{css} shipped with \lambtex offers an additional customisation options
for pictures.  As its name suggests, the classname style parameter '\mono{frame}' adds a
small framed shadow to the picture.  It is off by default, and esthetically it is more
sensible for rectangular pictures such as photographs.  In addition, the floatation
classnames '\mono{center}', '\mono{left}', and '\mono{right}' may also be used.

It is a common practice to add a numbered caption to a picture.  To achieve this in
\lambtex you must wrap the \mono{picture} block inside a \mono{figure}, as described
in \sref{sec:wrappers}.


\section[sec:subpage]{Sub-pages}

Another interesting \lambtex feature is the ability to display child documents
inside a framed block.  These nested documents, termed \emph{sub-pages}, are offered
by the \mono{subpage} environment.  While in general their usage is very limited and
it is unlikely you will ever need to use sub-pages, for some applications --- namely
writing manuals about \lambtex --- they are indispensable.

Sub-pages are not completely independent documents.  You may refer to elements inside
a sub-page from the main document and vice-versa.  Moreover, any notes and bibliography
entries (see \mbox{Sections \dref{sec:onnotes} and \dref{sec:onbiblio}}) declared inside a
sub-page are still associated with the main document.  Finally, take into consideration
that \lambtex will not automatically assign an ordering for sections and wrapper blocks
declared inside sub-pages (these blocks are part of the advanced language features
and are discussed in \sref{sec:headings} and \sref{sec:wrappers}, respectively).

You have already encountered several examples of sub-pages while reading this document,
as they have been used to illustrate the output of the composer in association with
each code sample.  The example below is the code that was used to produce the very first
sub-page you have encountered in this document, back in the \mref{sec:paragraphs}{section
on paragraph blocks}.

\begin<boxed>{source}
\begin{subpage}
This is the first example of a paragraph.  Note that
it is composed of inline text.

Lines can be broken at any point, because a paragraph
is only terminated by a blank line. \br
It is however possible to manually insert a line break
without terminating the paragraph.

\p{There is also a simple command that allows the
explicit declaration of a paragraph.}
\end{subpage}
\end{source}


\section[sec:headings]{Parts and sections}

\lambtex supports three different levels of sectioning: sections, sub-sections, and
sub-sub-sections.  The simple commands used for declaring each one are \mono{\\section},
\mono{\\subsection}, and \mono{\\subsubsection}, respectively.  These are automatically
ordered by \lambtex.  If you want to declare an unordered section, you must use an empty
ordering parameter.  Remember that ordering parameters are usually optional, and are
declared by enclosing between parenthesis the order you want to attribute. An empty order
simply means that only the opening and closing parenthesis should be present, without
any characters between them.  (Sectional commands declared inside sub-pages do require
an explicit ordering parameter, empty or not; check \sref{sec:order} for details).

In addition to sections, \lambtex documents may also be divided into \emph{parts}.
Parts are a higher-level division than sections, and are typically only used in
fairly complex documents.  This manual, for example, uses parts for making a clear
distinction between the language tutorial and the appendix sections.  Note that a new
part is declared with the homonymous simple command \mono{\\part}, and that it does
not reset the section counter.  In similarity to sections, parts can also be unordered.

Another common requirement associated with sectioning is the ability to refer to a given
part or section in a document.  With that in mind, \lambtex allows you to label a given
part or section declaration, and as described in the \mref{sec:intlinks}{following
section}, provides also a set of commands that create hyperlinks to a given label.
The label parameter is always optional and is enclosed in square brackets:

\begin<boxed>{source}
\part{Prelude}

\section{Introduction}

For more details, please check Section \dref{sec:details}.

\subsection[sec:details]{This is a sub-section}

\subsubsection(){This is an unordered sub-sub-section}

The quick brown fox jumps over the lazy brown dog.
\end{source}

The output produced by the \lambtex composer is as follows:

\begin{subpage}
\part(1){Prelude}

\section(1){Introduction}

For more details, please check Section 1.1.

\subsection(1.1)[sec:details]{This is a sub-section}

\subsubsection(){This is an unordered sub-sub-section}

The quick brown fox jumps over the lazy brown dog.
\end{subpage}

To conclude, and since this is our first encounter with label parameters, the reader
should note that labels have two constraints.  First, the starting character must be
a lowercase Roman letter.  Second, the allowed character set for subsequent characters
is limited to lowercase Roman letters, digits, and the characters '\mono{:}' (colon),
'\mono{-}' (dash), and '\mono{_}' (underscore).  Also note that all labels share
the same namespace; we recommend adopting the \caps{LaTeX} custom of using a prefix
indicating the target type.  In the example above, the prefix \mono{sec:} is used for
the sectioning family of commands.


\section[sec:intlinks]{Internal hyperlinks}

\lambtex supports hyperlinks both to external resources and into targets in
the same document.  The former is provided by the already described simple command
\mono{\\link}, whereas the latter relies on a set of different simple commands, each
one tailored for a different use and/or kind of target.  \sref{tab:hyperlinks} lists
all available hyperlinks commands, and the following source illustrates their use:

\begin<boxed>{source}
You can easily insert a link into an external resource such
as \link{http://www.wikipedia.com/}{Wikipedia}.  Internal
links come into several guises: they can be references to
notes\see{note:meta}; bibliographic references like those
available in \caps{LaTeX}\cite{bib:latex}; pointers to
labelled figures, tables, printouts, equations, or sections
such as \mbox{Section \dref{sec:anatomy}}; smart references to
those same elements such as \sref{sec:anatomy}; and finally,
references where the order is not automatically added (for
example, you may refer yet again \mref{sec:anatomy}{to that
same section}).
\end{source}

The corresponding output of the composer is the following:

\begin{subpage}
You can easily insert a link into an external resource such
as \link{http://www.wikipedia.com/}{Wikipedia}.  Internal
links come into several guises: they can be references to
notes\see{note:meta}; bibliographic references like those
available in \caps{LaTeX}\cite{bib:latex}; pointers to
labelled figures, tables, printouts, equations, or sections
such as \mbox{Section \dref{sec:anatomy}}; smart references to
those same elements such as \sref{sec:anatomy}; and finally,
references where the order is not automatically added (for
example, you may refer yet again \mref{sec:anatomy}{to that
same section}).
\end{subpage}

\begin[note:meta]{note}
This is a meta-note, a note about notes.  It is used to demonstrate the declaration
of notes.
\end{note}

Bear in mind that the use of the smart references offered by command \mono{\\sref}
is strongly recommended over the manual references provided by command \mono{\\dref}.
The reason is three-fold.  First, because smart references ensure the consistent use
of one naming scheme over an entire document and across several documents.  Second,
because smart references automatically make sure that a non-breaking space is inserted
between the element name and its ordering number, thus preventing their separation
by a line break (to achieve the same effect with manual references, the user must
explicitly place both elements inside an \mono{\\mbox} command, or use the \mono{\&nbsp;}
\caps{html} entity).  Third and last, because smart references will make a hyperlink
of both the name and the ordering, which is friendlier for mouse clicking and arguably
esthetically more pleasant.

On a related note, we recommend that \mono{\\see} and \mono{\\cite} commands not be
separated from the preceding element by spaces.  The default \caps{css} stylesheet
shipped with \lambtex already adds a small margin, and by not having spaces you
avoid the split across line breaks or the need to resort to the \mono{\\mbox} command
or the \mono{nbsp} \caps{html} entity.

Finally, though \lambtex offers no explicit support for subfigures, links towards them
are accommodated via the optional inline sequence accepted by commands \mono{\\dref}
and \mono{\\sref}.  This inline sequence is appended to the link text, thus allowing
you to specify a subfigure identifier (such as '\emph{a}', '\emph{b}', etc) when making
a link to the subfigure.

\begin[tab:hyperlinks]{table}
{The various hyperlinking commands available in \lambtex.  Note that the
\mono{\emph{uri}} and \mono{\emph{label}} sequences are interpreted as raw text.}
\begin<cols=rl>{tabular}
\head
| Command                                       | Description |
\body
| \mbox{\mono{\\link\{\emph{uri}\}}}            | Inserts a link to an external resource. |
| \mbox{\mono{\\link\{\emph{uri}\}\{\emph{inline}\}}}   | Alternate form of \mono{\\link}, which displays the inline sequence instead of the \caps{uri}. |
| \mbox{\mono{\\see\{\emph{label}\}}...}        | Refers to a document note (see \sref{sec:onnotes}). |
| \mbox{\mono{\\cite\{\emph{label}\}}...}       | Cites a bibliographic reference (see \sref{sec:onbiblio}). |
| \mbox{\mono{\\dref\{\emph{label}\}}}          | Inserts a link to a numbered section, part, equation, printout, table, figure, boxout, or theorem.  The text under the hyperlink is the order of the reference. |
| \mbox{\mono{\\dref\{\emph{label}\}\{\emph{inline}\}}} | Alternate form of \mono{\\dref}, which appends the inline sequence to the link text. Useful for subfigures, for example. |
| \mbox{\mono{\\sref\{\emph{label}\}}}          | Inserts a \emph{smart} reference to numbered section, part, equation, printout, table, figure, boxout, or theorem.  Besides the order, smart references also include the name of the kind of reference they are linking to. |
| \mbox{\mono{\\sref\{\emph{label}\}\{\emph{inline}\}}} | Alternate form of \mono{\\sref}, which appends the inline sequence to the link text. Useful for subfigures, for example. |
| \mbox{\mono{\\mref\{\emph{label}\}\{\emph{inline}\}}} | Inserts a reference to a section, part, equation, printout, table, figure, boxout, or theorem, but without including the order.  Instead, the description provided as the second mandatory parameter is used for the hyperlink. |
\end{tabular}
\end{table}


\section[sec:pull]{Pull-quotes}

\begin{pull}{Homer Simpson}
Weaseling out of things is important to learn:
it's what separates us from the animals.
Except the weasel.
\end{pull}

In typography, the special quotations such as the one above are termed \emph{pull-quotes}
or \emph{lift-out quotes}.  Pull-quotes serve different purposes than the regular
quotation blocks introduced in a \mref{sec:quotations}{previous} section.  Whereas
quotation blocks are almost exclusively used to quote a sizeable excerpt from an external
source, pull-quotes are either used to make a brief and named quotation such as the
one that opens this section, or to emphasize a passage from the very same document
where they are defined.

This block type is declared using the \mono{pull} environment command, and accepts an
optional primary parameter consisting of an inline sequence specifying the authorship
of the quote.  Therefore, to produce a pull-quote like the one that opens this section,
you can simply provide the following code:

\begin<boxed>{source}
\begin{pull}{Homer Simpson}
Weaseling out of things is important to learn:
it's what separates us from the animals.
Except the weasel.
\end{pull}
\end{source}

In similarity to their regular counterparts, the usefulness of pull-quotes carries with
it a warning against abuse.  Though one may be tempted to make heavy use of this feature,
bear in mind that its effectiveness goes down as their usage is increased.  Moreover,
it tends to make more sense for longer documents, and/or when it is floated to the left
or right.  For the latter reason, the default \caps{css} shipped with \lambtex allows
for the style parameters of a pull-quote to declare one of the previously discussed
floatation specifiers.


\section[sec:boxouts]{Declaring and using boxouts}

\newboxout{warning}{Warning}
\begin<right>{warning}{Boxout overload}
Boxouts are like quotation blocks:
they are most effective when not abused!
\end{warning}

Boxout blocks allow you emphasise a document fragment by placing it inside a box.
Boxouts are useful for warnings, notices, tips, reminders, etc.  Boxouts are custom
environments, in the sense that the command name that appears in the begin/end pairs
is not built-in and is instead defined in the document itself.  Therefore, before a
boxout is used, it must first be declared via the simple command \mono{\\newboxout}.
Moreover, boxouts are optionally titled and may be given an automatic ordering.

The command \mono{\\newboxout} accepts at least one, and up to three primary parameters.
The first parameter is the name of the boxout environment; this is the same name that
must subsequently be used inside the begin/end pairs.  Obviously, you may not name a
boxout with the same designation of an existing command, or with a designation that
begins with the name of an environment that uses the \caps{heredoc} technique (namely
\mono{verbatim}, \mono{source}, \mono{mathtexblk}, and \mono{mathmlblk}). Moreover, the
name must begin with a lowercase Roman letter and be composed only of lowercase Roman
letters, digits, and the character '\mono{_}' (underscore).  The second parameter,
if present, is an inline sequence indicating a fixed title that will appear in all
boxouts of this name.  Finally, the third parameter may be used to assign a counter to
this boxout.  If not present, the boxout is unnumbered; moreover, different boxouts
may share the same counter.  Note that the third parameter may only be given for
boxouts that do have a title, and thus also a second parameter.  In addition, bear
in mind that counter identifiers must follow the same rules laid out for labels,
as described in \sref{sec:headings}.  Counter identifiers do nevertheless reside in
their own separate namespace, independent of that of labels.

Subsequent to its declaration, using a new boxout named \mono{\emph{name}} is
simply a matter of wrapping its contents between \mono{\\begin\{\emph{name}\}} and
\mono{\\end\{\emph{name}\}}.  The begin tag accepts an optional primary parameter with an
extra title to assign to the boxout.  If the \mono{\\newboxout} boxout declaration did
not include a title, then this extra title becomes the sole title for this particular
instance of the boxout.  If on the other hand \mono{\\newboxout} already declared a
generic title, then the extra title is appended to the generic title.  Should neither
a generic nor an extra title be given, then the boxout will be titleless.

Suppose thus you wished to display a warning boxout like the one that opens this section.
It makes sense to name it \mono{warning}, and for each instance of the boxout to include
the generic title \emph{``Warning''}.  Moreover, you do not wish for warnings to be
numbered, so you should not assign a counter to it.  As for this particular instance
of the boxout, you wish the extra title \emph{``Boxout overload''} to be appended to
the generic title (note that the colon separating the generic and instance-specific
titles is automatically inserted as needed, so you do not need to include it manually).
 All these considerations can be realised with the following code:

\begin<boxed>{source}
\newboxout{warning}{Warning}

\begin{warning}{Boxout overload}
Boxouts are like quotation blocks:
they are most effective when not abused!
\end{warning}
\end{source}

At this point you may be wondering where does the customised warning sign come from.  Its
origin is simple: for each boxout named \mono{\emph{name}}, the \caps{html} output driver
from the \lambdoc library will include the classname \mono{doc_boxout_env_\emph{name}}
in the \caps{div} element that encapsulates the boxout.  The \caps{css} supplied
with the library takes advantage of this metadata, and contains sample customisation
for boxouts named \mono{warning} and \mono{tip}.  This trick provides an easy way to
customise your own boxouts.  The default \caps{css} also allows for boxouts to use
the floatation classname specifiers described in \sref{sec:verbatim}.

To conclude note that boxouts may be labelled, and that you may request the supression
of ordering for individual instances of numbered boxouts.  The latter is achieved in
the same manner as with sections and parts: by providing the empty ordering declaration
'\mono{()}'.  Labelled boxouts may be referenced from anywhere in the document via the
simple commands \mono{\\dref}, \mono{\\sref}, or \mono{\\mref} (the two former commands
may only be used for numbered boxouts, obviously).


\section[sec:theorems]{Theorem-like environments}

Theorems and their kin such as corollaries, lemmas, etc, are a staple of mathematical
and computer-science writing.  For this reason, \lambtex also supports them.
In practice, theorem-like environments are declared and used in pretty much the same
way as boxouts.  The major difference is that the declaration of a new theorem is
done via the simple command \mono{\\newtheorem}.  Moreover, theorems must always
have a title, and therefore the declaration has two mandatory primary parameters.
Similarly to boxouts, \mono{\\newtheorem} may optionally include a third primary
parameter specifying the counter if you wish the theorem to be numbered.  Note that
even though the same counter may be shared between multiple theorem-like environments,
you may not share a counter between a boxout and a theorem.  Because all counters
share the same namespace, you must make sure collisions do not occur.

Suppose thus you wish to create a theorem named \mono{theorem}, and whose title is,
unsurprisingly, \emph{``Theorem''}.  Moreover, you want the theorem to be numbered,
and its counter to be called '\mono{cnt:theorem}'.  One particular instantiation of
this theorem has the extra title \emph{``John McAuthor''}, reflecting its authorship.
This is the \lambtex source you may use:

\begin<boxed>{source}
\newtheorem{theorem}{Theorem}{cnt:theorem}

\begin{theorem}{John McAuthor}
Let an egg be defined as an ellipsoid whose radii across the $$x$$,
$$y$$, and $$z$$ axes are respectively $$a$$, $$b$$, and $$c$$.
Furthermore, the egg satisfies the scalene condition, and therefore
$$a > b > c$$.  All such eggs are optimally opened by their small-end.
\end{theorem}
\end{source}

The resulting output is shown below.  Note that in theorems the extra title is displayed
between parentheses, in accordance to convention.  Should you wish to customise the
\caps{css} for particular classes of theorems, you may make use of the same technique
described for boxouts.  Note, however, that for a theorem named \mono{\emph{name}},
the associated classname is \mono{doc_theorem_env_\emph{name}}.

\begin{subpage}
\newtheorem{theorem}{Theorem}{cnt:theorem}

\begin(1){theorem}{John McAuthor}
Let an egg be defined as an ellipsoid whose radii across the $$x$$,
$$y$$, and $$z$$ axes are respectively $$a$$, $$b$$, and $$c$$.
Furthermore, the egg satisfies the scalene condition, and therefore
$$a > b > c$$.  All such eggs are optimally opened by their small-end.
\end{theorem}
\end{subpage}


\section[sec:wrappers]{Wrappers}

Consider the previously discussed math blocks, source-code listings, tabular
environments, picture blocks, verbatim environments, and subpages.  All of these are
anonymous blocks, in the sense they do not have an associated label and therefore
cannot be referenced from elsewhere in the document.   It is however common for one
to wish to reference any of these block types from elsewhere in the document and/or
to provide them with a caption.  with this in mind, \lambtex offers a special kind
of block known as a \emph{wrapper}, which as its name implies, is essentially just
a wrapper around any of the above listed \emph{wrappable} blocks.  There are four
different classes of wrappers available:

\begin{description}

\item{Equations}

    These are declared via the environment command \mono{equation}, and
    may be used for wrapping \mono{mathtexblk} and \mono{mathmlblk} blocks.

\item{Printouts}

    These are used for wrapping \mono{source} environments, and are
    declared with the environment command \mono{printout}.

\item{Tables}

    Tables may only be used for wrapping \mono{tabular} environments,
    and are declared with the environment command \mono{table}.

\item{Figures}

    Finally, figures are declared with the environment command
    \mono{figure}, and may be used for wrapping \mono{verbatim},
    \mono{picture}, and \mono{subpage} blocks.

\end{description}

All these classes have two features in common: they may all be captioned, and each
class has an associated counter that is automatically incremented whenever a new
ordered wrapper is declared.  Note also that the wrapper itself may be labelled, as
illustrated by the code below, which shows the use of a \mono{figure} wrapper around
a \mono{picture} block:

\begin<boxed>{source}
\begin[fig:example]{figure}{This is the figure's caption.}
\picture{circle.png}{This is a circle}
\end{figure}
\end{source}

You can refer to the above declared figure from anywhere in the document by using the
label '\mono{fig:example}' with any of the simple commands \mono{\\dref}, \mono{\\sref},
and \mono{\\mref}.  The resulting figure (\sref{fig:example}) is shown below:

\begin[fig:example]{figure}{This is the figure's caption.}
\picture{circle.png}{This is a circle}
\end{figure}

Remember that captioning is optional.  When present, the caption is an inline sequence
declared as a primary parameter of the wrapper environment.  Should it not be given,
the counter associated with the wrapper is displayed between parentheses below the
wrapped block.  However, note that a caption-less wrapper must have an associated order,
and therefore if you declare the wrapper to have an empty ordering via the '\mono{()}'
specification, then you must provide it with a caption.  This restriction is in place
so that every wrapper has some form of context, either in the form of an ordering or
a caption.

It is perfectly reasonable and common for wrapped blocks to be floated on the sides
of the page.  Therefore, the default \caps{css} shipped with \lambtex allows for
the floatation specifiers '\mono{center}', '\mono{left}', and '\mono{right}' to be
provided as style parameters to any wrapper.  Note that with the exception of picture
and verbatim environments, with the default \caps{css} one can only float wrappable
blocks when they are actually placed inside a wrapper.  The rationale was discussed
in \sref{sec:verbatim}: in brief, floating a block is predicated on the block carrying
its own context, and thus being safely removable from the main flow of the document.

Note that context-carrying blocks are not tied to a precise location of the document
(that is, it won't matter much if they are placed a couple of paragraphs above or below,
as long as they can be found in the relevant section), and they have a fair degree
of independence from the contents in the main flow.  It is for this reason that you
may not directly float, say, a \mono{source} block.  It carries no context by itself,
and thus must be integrated into the main flow, or wrapped inside a \mono{printout}
whose caption and/or ordering provide the needed context.  Theorems --- even though
they have some independence and provide their own context --- are excluded from the
set of blocks allowed to float.  The rationale stems mainly from customary usage,
where theorems are integrated into the main flow of the document.


\section[sec:frontmatter]{Frontmatter: title, abstract, and table of contents}

This section addresses three semantic elements associated with large manuscripts
and which are placed before the main body of the document, at the so-called
\emph{frontmatter}.  These elements are titles and sub-titles, the abstract, and the
table of contents.

Titles in \lambtex are declared via the aptly named \mono{\\title} command.
This is a simple command that takes an inline sequence as parameter.  Closely related
is the also simple command \mono{\\subtitle}.  It is typically used for showing the
document's author, date, or actual sub-titles.

\caps{LaTeX} users are familiar with a richer set of sub-titles that includes special
commands for specifying the author and date.  There is obviously a trade-off between
increasing language complexity and improving semantic richness.  On this continuum,
\lambtex sits further along than plain \caps{TeX} on the direction towards semantic
richness.  However, the awareness of increased language complexity is ever-present,
and weights heavily on decisions such as this one.  It is for this reason that the
language offers a catch-all \mono{\\subtitle} command instead of multiple commands
tailored to each particular application.

Abstracts are commonly found on scientific papers, and \lambtex supports them as
well.  An abstract is declared by enclosing one or more embeddable blocks within an
\mono{abstract} environment.  Please consult \sref{sec:structure} for more details on
what constitutes a \emph{embeddable block}, and note the source fragment below for an
example of the various frontmatter blocks.

\begin<boxed>{source}
\title{Small-endian vs big-endian: the solution}
\subtitle{John McAuthor}
\subtitle{February 2009}

\begin{abstract}
This paper presents an irrefutable solution to the age-old question
of whether eggs should be opened by the small-end or the big-end.
Many insightful arguments shall be presented. (...)
\end{abstract}
\end{source}

This is the result produced by the \lambtex composer:

\begin{subpage}
\title{Small-endian vs big-endian: the solution}
\subtitle{John McAuthor}
\subtitle{February 2009}

\begin{abstract}
This paper presents an irrefutable solution to the age-old question
of whether eggs should be opened by the small-end or the big-end.
Many insightful arguments shall be presented. (...)
\end{abstract}
\end{subpage}

Finally, you can request that a table of contents be printed by issueing the
parameterless simple command \mono{\\toc}.  This can be done at any place in the
document, though convention and reason suggest that it be placed at the very beginning.
Note that even though \mono{\\toc} lacks any primary parameters (which in \lambtex are
always enclosed in curly brackets), it does accepts an optional label parameter.  This is
useful for referencing the table of contents from anywhere in the document.  Also, bear
in mind that the table of contents itself does not appear in the table of contents!
You may verify this fact in this manual's own \mref{sec:toc}{table of contents}.

\section{Rules}

A typographic rule is commonly used in fiction pieces to clearly separate unrelated
paragraphs.  For the sake of completeness, \lambtex supports also this feature.
You may declare it using the \mono{\\rule} simple command, or its synonym \mono{\\hr}.
These commands are completely parameterless, as the code sample below demonstrates:

\begin<boxed>{source}
This is a paragraph.

\rule

This is another paragraph.
\end{source}

The generated \caps{xhtml} result depends of course on the installed \caps{css}.
The default stylesheet produces the following result:

\begin{subpage}
This is a paragraph.

\rule

This is another paragraph.
\end{subpage}


\section[sec:onnotes]{Adding notes to a document}

Markup languages such as \caps{LaTeX}, whose roots stem from page-oriented designs,
have the concept of a \emph{footnote}.  \lambtex is targeted at web publishing,
a domain where the concept of a document being composed of individual pages is not
meaningful.  For this reason, \lambtex's equivalent concept is that of notes
attached to the end of the document.

A note is declared via the environment command \mono{note}.  Notes can of course be
labelled, and may be referenced from anywhere in the document via the simple command
\mono{\\see}.  The latter takes at least one primary parameter, and therefore you may
reference multiple notes simply by providing additional primary parameters.

One important feature to remember is that the location where the note is declared is
\bold{not} the same as where it will be appear in the rendered document (for this reason,
notes are termed \emph{ghost blocks}).  Instead,  all notes will be printed together
wherever the simple commands \mono{\\notes} or \mono{\\rawnotes} are declared (the sole
difference between the two is that \mono{\\notes} will also create an eponymously named
unordered section, just like \mref{sec:thenotes}{the one present} in this manual, whereas
\mono{\\rawnotes} simply dumps a listing of the document's notes, without any heading).
Note that the ``ghostly'' character of note blocks allows them to be declared anywhere
in the document, a handy feature if you wish to keep a note declaration close to the
place where it is declared.

The sample code below illustrates a note being referenced inside a paragraph and being
defined afterwards.  The example concludes with the simple command \mono{\\notes}
requesting the printing of the eponymous section together with the defined note.

\begin<boxed>{source}
This is a normal\see{note:example} paragraph that references
an end-document note.  You may read all notes declared in the
document \mref{sec:notes}{in this section}.

\begin[note:example]{note}
And by 'normal' we mean 'not normal'.
\end{note}

\notes[sec:notes]
\end{source}


\section[sec:onbiblio]{Adding a bibliography}

Similarly to notes, bibliographic entries are also ghost blocks.  Therefore, you may
declare them anywhere in the document, and they will only be printed when you issue
either of the simple commands \mono{\\bibliography} or \mono{\\rawbibliography}.
The former version is preferred, as it automatically takes care of creating also a
section heading.  The latter version simply dumps the contents of the bibliography.

Each individual bibliographic entry can take either of two forms: a long form or a
short form.  The long form is contained within an environment command \mono{lbib},
which itself accepts three sub-blocks, defined by the simple commands \mono{\\who},
\mono{\\what}, and \mono{\\where}.  These blocks are used for defining the author,
the title, and the location (journal, proceedings, web page) of the bibliographic
reference, respectively.  The short form is defined by the simple command \mono{\\sbib},
which takes inline text as the mandatory parameter.

The code below illustrates the declaration of two bibliographic entries, one long and
one short.  Note that the entries can be referenced by the simple command \mono{\\cite}.
Also, and in similarity to the command \mono{\\see}, it is possible to reference multiple
entries with a single invocation of \mono{\\cite}, since this command accepts multiple
primary parameters.

\begin<boxed>{source}
Lambtex is largely inspired by LaTeX\cite{bib:latex},
and is available in the Lambdoc library\cite{bib:lambdoc}.

\begin[bib:latex]{lbib}
\who{Leslie Lamport}
\what{LaTeX: a Document Preparation System (2\sup{nd} edition)}
\where{Addison-Wesley Professional, 1994, ISBN 0-201-52983-1}
\end{lbib}

\sbib[bib:lambdoc]{\link{http://lambdoc.forge.ocamlcore.org/}}
\end{source}

Whereas \caps{BibTeX}\cite{bib:bibtex} offers a very rich set of tags for bibliographic
entries (author, title, journal volume, page numbers, conference location, etc), \lambtex
limits this set to only three main tags, even when using the long form \mono{lbib}.
Those familiar with \caps{BibTeX} may feel constrained by this paucity.  However,
one must take into account that \lambtex was designed with the web as a foremost
application area.  In this context, one can have satisfactory --- in the sense that
they are readily accessible --- bibliographic entries based solely on a hyperlink.
The long form's \mono{\\where} command can therefore be used for linking to an external
web resource, or if not available online, to provide all the miscelaneous information
required to locate the resource in ``dead tree world''.  In fact, most bibliography
needs in the web domain are satisfied completely by the short form \mono{\\sbib}.


\section[sec:onappendix]{Creating an appendix}

Large document typically have fragments that do not fit into the main flow.
In many cases, these fragments can satisfactorily be included as notes (remember
\sref{sec:onnotes}).  In other situations, however, it may make more sense to
attach them as appendices to the main document.  In \lambtex, creating an appendix
is simply a matter of issueing the simple parameterless command \mono{\\appendix}.
Besides declaring an unordered part of the same name, this command also marks all
subsequent sections as belonging to the appendix.  This means that an appendiced
section is declared the usual way, via the command \mono{\\section} (the same applies
to sub-sections and sub-sub-sections, of course).  The only distinguishing aspect that
makes it an appendiced section is being declared after \mono{\\appendix}.

You can see an example of appendiced sections in this very manual.  Note how all
sections declared after the \mref{part:theappendix}{appendix marker} are automatically
ordered using alphabetic characters instead of numbers, as per the usual convention.
Smart references produced by command \mono{\\sref} automatically adjust as well ---
yet another reason why you should avoid its ``dumb'' counterpart \mono{\\dref}.


\section[sec:macros]{Macros}

\lambtex offers limited macro functionality.  Macros provide simple textual replacement,
possibly parameterised.  Note that these are not full-blown programming constructs,
being instead aimed at reducing boilerplate and ensuring consistent formatting of
common text patterns.

Before being used, a macro must be declared via the simple command \mono{\\newmacro}.
This command takes three mandatory primary parameters: the name of the macro, the number
of arguments, and an inline sequence containing the text of the macro.  Macro names
must obviously be unique and not overlap with the names of built-in commands, but they
live in their own private namespace.  Moreover, and similarly to boxouts and theorems,
the name must begin with a lowercase Roman letter and be followed by any combination
of lowercase Roman letters, digits, and the character '\mono{_}' (underscore).

Should the macro take any arguments, these can be referenced from inside the macro
text by the alias \mono{\\arg\{\emph{num}\}}, where the \mono{\emph{num}} between curly
brackets is an integer indicating the argument number.  Obviously, a macro with $$n$$
arguments may not reference an argument $$n+1$$ or higher.  Moreover, arguments are
expanded as inline sequences.

Wherever an inline sequence is accepted, a macro named \mono{name} can be invoked via
the simple command \mono{\\name}.  The invocation of the macro causes the macro's text
to be inserted at invocation point.  Should the macro have any arguments, these must
follow the macro's name immediately, being each one bracketed individually.  Therefore,
the invocation of macro \mono{name} with arguments \mono{inline1} and \mono{inline2}
is done with \mbox{\mono{\\name\{inline1\}\{inline2\}}}.  Note that because macros
can be invoked wherever an inline sequence is accepted and the text of a macro is
an inline sequence itself, it follows that macros may invoke other macros.  However,
because a macro must be declared before being invoked, mutual recursion is not allowed.
Moreover, a macro may not invoke itself either.  These restrictions are in place to
ensure that macro expansion always terminates.

While token concatenation with parameterised macros is straightforward, you may
wonder how it can be achieved with parameterless macros.  To illustrate this
issue, suppose you have defined a macro \mono{super} that expands to the text
\emph{``super''}.  Now, you wish to concatenate the invocation of this macro with the
text \emph{``market''}.  However, entering \mono{\\super market} produces an unwanted
space, whereas \mono{\\supermarket} produces an error, since the macro \mono{supermarket}
is undefined.  Note that if the macro had any arguments this problem would be moot,
since the brackets naturally delimitate the macro name.  Therefore one solution is
to always declare macros with at least one parameter, even if it is a dummy that can
be discarded.  There is however a better alternative, one which takes advantage of
the fact that all commands accept (at least) an empty sequence as extra parameters:
simply put \mono{<}&#8202;\mono{>} after the macro name.  Concatenation in this example
can thus be achieved by entering \mbox{\mono{\\super<}&#8202;\mono{>market}}.

Let us now consider a full example of macro declaration.  We wish to define a macro
\mono{author} that when invoked expands to the full name of an author.  We also
define a macro named \mono{greet} that takes two arguments and produces a greeting.
These two macros are then used in a small sample document.  This is the \lambtex source:

\begin<boxed>{source}
\newmacro{author}{0}{John McAuthor}
\newmacro{greet}{2}{Greetings \bold{\arg{1}}, and welcome to \arg{2}}

It was then that \author reached his destination.  The gatekeeper said,
``\greet{\author}{this land}'', and there was much rejoicing.
\end{source}

And this is the resulting output:

\begin{subpage}
\newmacro{author}{0}{John McAuthor}
\newmacro{greet}{2}{Greetings \bold{\arg{1}}, and welcome to \arg{2}}

It was then that \author reached his destination.  The gatekeeper said,
``\greet{\author}{this land}'', and there was much rejoicing.
\end{subpage}

Note that the arguments to a macro are inline text, and thus it is possible
for the argument to consist of the invocation of a macro.  On the other hand,
macros cannot be used to replace non-inline sequences of a command.


\section{Concluding remarks}

You are now familiar with the all the features available in the \lambtex markup language.
We hope you will find it rich enough to cover your web publishing needs, while still
being simple enough to dispense a significant investment of time learning it.

We advise that you peruse the appendices attached to this manual.  There you can find
some reference material that did not fit into the flow of the main body of the document.
\sref{sec:structure}, for example, discusses at length the structure of the \lambtex
documents, while \sref{sec:synonyms} summarises all command synonyms available,
and \sref{sec:parameters} covers all you need to know concerning optional arguments.
Finally, unless you are writing about \lambtex itself, there is little chance you will
need the information contained in \sref{sec:order}.  But if you do, it focuses on the
somewhat arcane aspects concerning numbering in sub-pages.


\appendix[part:theappendix]


\section[sec:structure]{The structure of \lambtex documents}

We described a \lambtex document as being composed of a list of blocks, with some
of these blocks being themselves composed of other blocks.  We have, however, glossed
over some of the restrictions imposed on this structure.  The goal of this section is
therefore to rectify that omission by providing a full account of the structure of a
\lambtex document.

Blocks come in different types.  Moreover, it is possible to establish a subtyping
hierarchy that relates the different block types. \sref{fig:hierarchy} illustrates that
hierarchy, showing block types inside rounded rectangles.  Concerning the subtyping
relation, an arrow pointing from a source block into a target block should be interpreted
as meaning that the source block is a subtype of the target, and dually, that the target
is a supertype of the source.  In practice, this means that a subtype block may be used
in any circumstance where its supertype is allowed.  However, the converse is not true.

\begin[fig:hierarchy]{figure}
{The subtyping hierarchy for \lambtex blocks.  An arrow pointing from a source
block to a target block indicates that the source is a subtype of the target and
is therefore accepted in any circumstances where the target would be accepted.}
\picture{hierarchy.png}{Hierarchy}
\end{figure}

Note the list attached to each of the block types, which indicates the actual
document blocks that belong to that type.  An \emph{embeddable block}, for example,
can be either a paragraph, an itemize, enumerate, or description list, a verse block,
a block containing mathematics, etc.  Notice that many blocks have a \emph{\mono{of}
...} declaration after their name; these blocks --- termed \emph{recursive blocks}
--- are those whose contents are other blocks.

If a \lambtex document were unconstrained, one would expect a recursive block to
be composed of a list of super blocks.  Because super blocks sit at the top of the
hierarchy, \bold{any} block can be used where a super block is allowed.  Therefore,
in an unconstrained scenario, it would be perfectly legal for a quotation block to
be the parent of a section, or for an abstract block to include another.  Needless to
say, an unconstrained document would accommodate for pathological situations bearing
no resemblance to any reasonable real-world documents.  It is for this reason that a
\lambtex document has a few constraints in place.  As you shall see, these constraints
are however fairly minor and still allow for very flexible documents.

In basic terms, documents are constrained by limiting the types of blocks accepted
as children of recursive blocks.  The exact subtype allowed for each recursive block
is shown in the figure as an \emph{\mono{of} ...} declaration after the block name.
Let us consider, for example, a quote block.  It only allows a list of \emph{quotable
blocks} as children.  This means that you cannot quote a figure nor an abstract, but
you can do so with any of the quotable blocks or subtypes (\emph{embeddable blocks},
namely).  Furthermore, these rules apply to the entire descendant tree.  This means
that if an itemize block is the child of a quote block, then the most restricted rules
of the parent apply also to the child's descendants.  In this case that would mean
that the itemize block would also only accept quotable blocks as children instead of
its usually more permissive allowance of any \emph{listable block}.  The exception to
this rule is the \mono{subpage} block, which resets the hierarchy.


\section[sec:synonyms]{Command synonyms}

For reference purposes, \sref{tab:synonyms} lists all existing synonyms for both simple
and environment commands (the former's name is prefixed by a backslash).  Remember that
in the case of environment commands, the same name must be used for the \mono{begin}
and \mono{end} tags, even if they are synonyms.

\begin[tab:synonyms]{table}{List of all available command synonyms.}
\begin<cols=rl>{tabular}
\head
| Canonical name            | Synonyms                      |
\body
| \mono{\\bold}             | \mono{\\strong}, \mono{\\b}   |
| \mono{\\mono}             | \mono{\\tt}                   |
| \mono{\\emph}             | \mono{\\em}, \mono{\\i}       |
| \mono{\\item}             | \mono{\\li}                   |
| \mono{\\link}             | \mono{\\a}                    |
| \mono{\\paragraph}        | \mono{\\p}                    |
| \mono{\\rule}             | \mono{\\hr}                   |
| \mono{\\section}          | \mono{\\h1}                   |
| \mono{\\subsection}       | \mono{\\h2}                   |
| \mono{\\subsubsection}    | \mono{\\h3}                   |
\body
| \mono{description}        | \mono{dl}                     |
| \mono{enumerate}          | \mono{ol}                     |
| \mono{itemize}            | \mono{itemise}, \mono{ul}     |
| \mono{verbatim}           | \mono{pre}                    |
\end{tabular}
\end{table}


\section[sec:parameters]{Command parameters}

The typical simple command in \lambtex has one or two primary parameters enclosed between
curly brackets.  In addition, many commands (both simple and environment) accept also
optional parameters that specify an ordering, a label, or any style information.
When present, optional parameters may be specified in any order, as long as they
precede any of the primary parameters.  Moreover, each class of optional parameter is
distinguished by the set of characters used for delimitation.  Ordering parameters
are enclosed in curved brackets, labelling parameters in square brackets, and style
parameters between angle brackets.

Note that not all optional parameters are meaningful for all existing commands.
This is especially the case for order and label parameters, which are available only
for some commands.  \sref{tab:permlist} lists the classes accepted for each of all
existing commands, both simple and environment ones.  An exact definition of the terms
used is shown in \sref{tab:permdesc}.

\begin[tab:permdesc]{table}
{Definition of the various permissions.  These terms are used in \sref{tab:permlist}.}
\begin<cols=cl>{tabular}
\head
| Permission    | Definition                                                    |
\body
| \optional     | The parameter is optional, but may not be empty if present.   |
| \mandatory0   | The parameter is mandatory, but it may be empty.              |
| \forbidden    | The parameter is altogether forbidden.                        |
| \forbidden0   | The parameter is forbidden, unless it is empty.               |
\end{tabular}
\end{table}

Note that it is always possible to specify empty extra parameters for any command.
This is done so that token concatenation is possible for simple commands that take
no primary parameters, as is often the case with user-defined macros.  A practical
application of this feature is described in \sref{sec:macros}.

\begin[tab:permlist]{table}
{The permissions associated with the various \lambtex commands for order and label
optional parameters.  Note that simple commands are shown prefixed with a backslash,
whereas environment commands are enclosed inside brackets.  Within each category,
commands are shown alphabetically.  Furthermore, the commands \mono{\emph{\\macro}},
\mono{\emph{boxout}} and \mono{\emph{theorem}} are shown in italics because the actual
command name is user-defined.}
\begin<cols=rll>{tabular}
\head
| Command                   | Order         | Label         |
\body
| \mono{\\answer}           | \forbidden    | \forbidden    |
| \mono{\\appendix}         | \forbidden    | \optional     |
| \mono{\\arg}              | \forbidden    | \forbidden    |
| \mono{\\bibliography}     | \forbidden    | \optional     |
| \mono{\\body}             | \forbidden    | \forbidden    |
| \mono{\\bold}             | \forbidden    | \forbidden    |
| \mono{\\br}               | \forbidden    | \forbidden    |
| \mono{\\caps}             | \forbidden    | \forbidden    |
| \mono{\\cite}             | \forbidden    | \forbidden    |
| \mono{\\mono}             | \forbidden    | \forbidden    |
| \mono{\\del}              | \forbidden    | \forbidden    |
| \mono{\\dref}             | \forbidden    | \forbidden    |
| \mono{\\foot}             | \forbidden    | \forbidden    |
| \mono{\\glyph}            | \forbidden    | \forbidden    |
| \mono{\\head}             | \forbidden    | \forbidden    |
| \mono{\\ins}              | \forbidden    | \forbidden    |
| \mono{\\item}             | \forbidden    | \forbidden    |
| \mono{\\link}             | \forbidden    | \forbidden    |
| \mono{\emph{\\macro}}     | \forbidden    | \forbidden    |
| \mono{\\mbox}             | \forbidden    | \forbidden    |
| \mono{\\mref}             | \forbidden    | \forbidden    |
| \mono{\\notes}            | \forbidden    | \optional     |
| \mono{\\paragraph}        | \forbidden    | \forbidden    |
| \mono{\\part}             | \depends0     | \optional     |
| \mono{\\picture}          | \forbidden    | \forbidden    |
| \mono{\\question}         | \forbidden    | \forbidden    |
| \mono{\\ranswer}          | \forbidden    | \forbidden    |
| \mono{\\rquestion}        | \forbidden    | \forbidden    |
| \mono{\\rule}             | \forbidden    | \forbidden    |
| \mono{\\section}          | \depends0     | \optional     |
| \mono{\\see}              | \forbidden    | \forbidden    |
| \mono{\\span}             | \forbidden    | \forbidden    |
| \mono{\\sref}             | \forbidden    | \forbidden    |
| \mono{\\sub}              | \forbidden    | \forbidden    |
| \mono{\\subsection}       | \depends0     | \optional     |
| \mono{\\subsubsection}    | \depends0     | \optional     |
| \mono{\\subtitle}         | \forbidden    | \forbidden    |
| \mono{\\sup}              | \forbidden    | \forbidden    |
| \mono{\\title}            | \forbidden    | \forbidden    |
| \mono{\\toc}              | \forbidden    | \optional     |
| \mono{\\what}             | \forbidden    | \forbidden    |
| \mono{\\where}            | \forbidden    | \forbidden    |
| \mono{\\who}              | \forbidden    | \forbidden    |
\body
| \mono{abstract}           | \forbidden    | \forbidden    |
| \mono{bib}                | \forbidden    | \optional     |
| \mono{\emph{boxout}}      | \depends0     | \optional     |
| \mono{decor}              | \forbidden    | \forbidden    |
| \mono{description}        | \forbidden    | \forbidden    |
| \mono{enumerate}          | \forbidden    | \forbidden    |
| \mono{equation}           | \depends0     | \optional     |
| \mono{figure}             | \depends0     | \optional     |
| \mono{itemize}            | \forbidden    | \forbidden    |
| \mono{mathmlblk}          | \forbidden    | \forbidden    |
| \mono{mathtexblk}         | \forbidden    | \forbidden    |
| \mono{note}               | \forbidden    | \optional     |
| \mono{quote}              | \forbidden    | \forbidden    |
| \mono{printout}           | \depends0     | \optional     |
| \mono{pull}               | \forbidden    | \forbidden    |
| \mono{qanda}              | \forbidden    | \forbidden    |
| \mono{source}             | \forbidden    | \forbidden    |
| \mono{subpage}            | \forbidden    | \forbidden    |
| \mono{table}              | \depends0     | \optional     |
| \mono{tabular}            | \forbidden    | \forbidden    |
| \mono{\emph{theorem}}     | \depends0     | \optional     |
| \mono{verbatim}           | \forbidden    | \forbidden    |
| \mono{verse}              | \forbidden    | \forbidden    |
\end{tabular}
\end{table}


\section[sec:order]{Notes on ordering}

This section applies to those blocks that may have an associated ordering, namely
wrapper blocks, sectional blocks, and the custom blocks (boxouts and theorems).
Recall that there are four elements in \lambtex that are categorised as wrappers:
equations, printouts, tables, and figures.  As for sectional blocks, these are the
ones that produce parts, sections, sub-sections, and sub-sub-sections.  There are
some special --- though as you shall see based on common sense --- rules that must be
followed concerning the order parameter and the hyperlinking of blocks with associated
ordering, particularly in the context of sub-pages.  The rules are as follows:

\begin{enumerate}

\item
    Only numbered elements may be referred to using \mono{\\dref} and \mono{\\sref}
    commands:

    \p<noindent>{Unless the user explicitly demanded an empty ordering, sectional
    commands and wrappers have by default a number.  In the case of custom
    environments, the situation varies.}

    It makes sense that the hyperlinking commands that are supposed to produce
    a number should only be used with numbered wrappers, numbered sectional
    blocks, or numbered custom environments.  These commands are \mono{\\dref}
    and \mono{\\sref}.  On the other hand, because \mono{\\mref} does not output
    a number, it is always allowed.

\item
    You must explicitly provide a manual ordering for sub-paged wrappers, sectional
    blocks, or custom blocks:

    \p<noindent>{A sub-paged element is one that is declared inside a \mono{subpage}
    block (see \sref{sec:subpage}).  Because they lie outside the main flow of the
    document, these elements are not automatically numbered.  You must therefore
    explicitly provide them with an ordering parameter by inserting the numbering
    between parentheses.  Specifying an empty ordering is of course also allowed.}

\item
    You may not provide a manual ordering for wrappers, sectional blocks, or custom
    blocks defined outside a sub-page:

    \p<noindent>{This rule is complementary to the previous one.  Because these
    blocks are numbered automatically by \lambtex when present in the main document,
    you may not interfere with that automatism by providing an explicit ordering.
    Note, however, that an empty ordering is always allowed.}

\end{enumerate}


\part(){Backmatter}


\notes[sec:thenotes]


\bibliography[sec:thebiblio]

\begin[bib:latex]{lbib}
\who{Leslie Lamport}
\what{LaTeX: a Document Preparation System (2\sup{nd} edition)}
\where{Addison-Wesley Professional, 1994. ISBN 0-201-52983-1.}
\end{lbib}

\sbib[bib:lambdoc]{\link{http://lambdoc.forge.ocamlcore.org/}}

\sbib[bib:mathml]{\link{bah}{http://www.w3.org/Math/}}

\begin[bib:tex]{lbib}
\who{Donald E. Knuth}
\what{The TeXbook (Computers and Typesetting, Volume A)}
\where{Addison-Wesley, 1984. ISBN 0-201-13448-9.}
\end{lbib}

\begin[bib:bibtex]{lbib}
\who{Oren Patashnik}
\what{BibTeXing}
\where{\link{blah}{http://www.ctan.org/get/biblio/bibtex/contrib/doc/btxdoc.pdf}}
\end{lbib}

