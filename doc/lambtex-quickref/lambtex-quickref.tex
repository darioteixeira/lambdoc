\documentclass[9pt]{extarticle}

\usepackage{geometry}                           % To define page size, margins, etc.
\usepackage[pdftex]{hyperref}                   % For hyper-references in the generated PDF.
\usepackage[hyperref]{xcolor}                   % To use colours in the document.
\usepackage[pdftex]{graphicx}                   % To import graphics.
\usepackage{epstopdf}                           % For on-the-fly conversion of eps to pdf.
\usepackage[british]{babel}                     % For British English hyphenation patterns.
\usepackage[latin1]{inputenc}                   % For Latin1 (ISO-8859-1) input encoding.
\usepackage[T1]{fontenc}                        % For T1 character encoding.
\usepackage{textcomp}                           % For T1 character encoding.
\usepackage{ae}                                 % For T1 encoding using Type 1 fonts.
\usepackage{mathptmx}                           % For Type 1 "Times Roman" serif text and math.
\usepackage[scaled=0.92]{helvet}                % For Type 1 "Helvetica" sans serif text.
\usepackage{courier}                            % For Type 1 "Courier" monospaced text.
\usepackage{xspace}                             % For automatic handling of spaces after macros.
\usepackage{booktabs}				% For better tables.
\usepackage{multirow}				% For merging rows.
\usepackage{soul}				% For strike-through.
\usepackage[titles]{tocloft}                    % For Table of Contents customisation.
\usepackage{subfig}                             % For subfigures.
\usepackage{fancyvrb}                           % For fancy verbatim environments.
\usepackage[stable]{footmisc}                   % For stable footnotes.
\usepackage{float}                              % For extra control over floats.
\usepackage{titlesec}                           % To configure sections headings.
\usepackage[nottoc]{tocbibind}                  % To put the Bibliography in the TOC.
\usepackage{units}                              % To typeset units.

\geometry{dvipdfm,a4paper,landscape,centering,tmargin=2cm,bmargin=2cm,lmargin=2cm,rmargin=2cm,nomarginpar}

\definecolor{ins}{rgb}{0.0,0.5,0.0}
\definecolor{del}{rgb}{0.5,0.0,0.0}

\definecolor{no}{rgb}{0.9,0.8,0.8}
\definecolor{opt}{rgb}{0,0.5,0}
\definecolor{dep}{rgb}{0.8,0.4,0}

\definecolor{rex}{rgb}{1.0,0.2,0.2}

\definecolor{composition}{rgb}{0,0.3,0}
\definecolor{manuscript}{rgb}{0,0,0.3}

\hypersetup{%
	linkcolor=RoyalBlue,
	urlcolor=RoyalBlue,
	plainpages=false,
	pdfpagelabels,
	hyperindex,
	colorlinks=true}

\hypersetup{%
	pdftitle={Lambtex Quick Reference},
	pdfauthor={Dario Teixeira (dario.teixeira@yahoo.com)},
	pdfsubject={lambtex},
	pdfkeywords={lambtex, lambdoc}}

\begin{document}

\newcommand{\no}[0]{\textcolor{no}{---}}
\newcommand{\opt}[0]{\textcolor{opt}{opt}}
\newcommand{\dep}[0]{\textcolor{dep}{dep\textsuperscript{1}}}
\newcommand{\depz}[0]{\textcolor{dep}{dep\textsuperscript{2}}}

\newcommand{\C}[0]{\textcolor{composition}{C}}
\newcommand{\M}[0]{\textcolor{manuscript}{M}}

\newcommand{\op}[1]{\textbf{#1}}
\newcommand{\hd}[1]{\textbf{#1}}
\newcommand{\ins}[1]{\textcolor{ins}{#1}}
\newcommand{\del}[1]{\textcolor{del}{#1}}
\newcommand{\simc}[1]{\textcolor{composition}{\texttt{\bfseries$\backslash$#1}}}
\newcommand{\simm}[1]{\textcolor{manuscript}{\texttt{\bfseries$\backslash$#1}}}
\newcommand{\envc}[1]{\textcolor{composition}{\texttt{\bfseries#1}}}
\newcommand{\envm}[1]{\textcolor{manuscript}{\texttt{\bfseries#1}}}
\newcommand{\textsubscript}[1]{\ensuremath{_{\textrm{#1}}}}

\newcommand{\param}[1]{\textsc{\{\emph{#1}\}}}
\newcommand{\inline}[0]{\param{inline}}
\newcommand{\thelabel}[0]{\param{label}}
\newcommand{\raw}[0]{\param{raw}}

\newcommand{\rexoper}[1]{\textcolor{rex}{\textbf{#1}}}
\newcommand{\rexgroup}[2]{\,\rexoper{[}#1\rexoper{]}\textsuperscript{\rexoper{#2}}\,}
\newcommand{\rexo}[1]{\rexgroup{#1}{?}}
\newcommand{\rexp}[1]{\rexgroup{#1}{+}}

\newcommand{\ex}[1]{\small\texttt{#1}}

\begin{tabular}{rlllllcl}

\toprule

			&				& \multicolumn{4}{c}{\hd{Parameters}}							& & \\

\cmidrule{3-6}\\

\hd{Command}		& \hd{Synonyms}			& \hd{Primary}				& \hd{Order}	& \hd{Label}	& \hd{Extra}	& \hd{T}
& \hd{Description}\\

\midrule

\simc{br}		& \no				& \no					& \no		& \no		& \no		& \C
& Inserts a line break within the same paragraph.\\

\simc{bold}		& \simc{b}, \simc{strong}	& \inline				& \no		& \no		& \no		& \C
& Sets the inline parameter in \textbf{bold} font.\\

\simc{emph}		& \simc{i}, \simc{em}		& \inline				& \no		& \no		& \no		& \C
& Sets the inline parameter in \emph{emphasised} font.\\

\simc{code}		& \simc{tt}			& \inline				& \no		& \no		& \no		& \C
& Sets the inline parameter in \texttt{monospaced} (teletype) font.\\

\simc{caps}		& \no				& \inline				& \no		& \no		& \no		& \C
& Sets the inline parameter in \textsc{small caps}.\\

\simc{ins}		& \no				& \inline				& \no		& \no		& \no		& \C
& The inline parameter indicates \ins{inserted} text.\\

\simc{del}		& \no				& \inline				& \no		& \no		& \no		& \C
& The inline parameter indicates \del{deleted} text.\\

\simc{sup}		& \no				& \inline				& \no		& \no		& \no		& \C
& Sets the inline parameter as \textsuperscript{superscript} text.\\

\simc{sub}		& \no				& \inline				& \no		& \no		& \no		& \C
& Sets the inline parameter as \textsubscript{subscript} text.\\

\simc{mbox}		& \no				& \inline				& \no		& \no		& \no		& \C
& Specifies that the inline parameter should not be split across lines.\\

\simc{span}		& \no				& \inline				& \no		& \no		& \ex{class}	& \C
& Definition of spans of inline text with a custom \textsc{xhtml} class name.\\

\simc{link}		& \simc{a}			& \raw\rexo{\inline}			& \no		& \no		& \no		& \C
& Creates a link to the specified \textsc{uri} using the inline text for display.\\

\simm{cite}		& \no				& \thelabel				& \no		& \no		& \no		& \M
& Creates a link to the specified entry in the bibliography.\\

\simm{see}		& \no				& \thelabel				& \no		& \no		& \no		& \M
& Creates a link to the specified document note.\\

\simm{ref}		& \no				& \thelabel				& \no		& \no		& \no		& \M
& Creates a manual reference to the specified element.\\

\simm{sref}		& \no				& \thelabel				& \no		& \no		& \no		& \M
& Creates an automatic reference to the specified element.\\

\simm{mref}		& \no				& \thelabel\inline			& \no		& \no		& \no		& \M
& Creates a reference to the specified element using the custom inline text for display.\\

\bottomrule

\end{tabular}

\begin{tabular}{rlllllcl}

\toprule

			&				& \multicolumn{4}{c}{\hd{Parameters}} 									& & \\

\cmidrule{3-6}\\

\hd{Command}		& \hd{Synonyms}			& \hd{Primary}				& \hd{Order}	& \hd{Label}	& \hd{Extra}			&\hd{T}
& \hd{Description}\\

\midrule

\simc{paragraph}	& \simc{p}			& \inline				& \no		& \no		& \ex{initial}, \ex{indent}	& \C
& Declares a paragraph of text.\\

\envc{itemize}		& \envc{ul}, \envc{itemise}	& \no					& \no		& \no		& \ex{bul}			& \C
& Declares an unordered list.\\

\envc{enumerate}	& \envc{ol}			& \no					& \no		& \no		& \ex{num}			& \C
& Declares an ordered list.\\

\envc{description}	& \envc{dl}			& \no					& \no		& \no		& \no				& \C
& Declares a description list.\\

\envc{qanda}		& \no				& \no					& \no		& \no		& \no				& \C
& Declares a Q\&A (interview) environment.\\

\envc{verse}		& \no				& \no					& \no		& \no		& \no				& \C
& Declares a verse block.\\

\envc{quote}		& \no				& \no					& \no		& \no		& \no				& \C
& Declares a quotation block.\\

\envc{mathtex}		& \no				& \no					& \no		& \no		& \no				& \C
& Declares a math block in \TeX format.\\

\envc{mathml}		& \no				& \no					& \no		& \no		& \no				& \C
& Declares a math block in \textsc{MathML} format.\\

\envc{source}		& \no				& \no					& \no		& \no		& \ex{box}, \ex{nums}, \ex{zebra}, \ex{lang}& \C
& Declares a verbatim block containing source-code.\\

\envc{tabular}		& \no				& \raw					& \no		& \no		& \no				& \C
& Declares a tabular environment.\\

\envc{verbatim}		& \envc{pre}			& \no					& \no		& \no		& \ex{mult}			& \C
& Declares a verbatim block with \textsc{ascii}-art.\\

\simc{image}		& \no				& \thelabel\raw				& \no		& \no		& \ex{frame}, \ex{width}	& \C
& Inserts an image.\\

\envc{subpage}		& \no				& \no					& \no		& \no		& \no				& \C
& Declares a sub-page containing a child document.\\

\envm{decor}		& \no				& \no					& \no		& \no		& \ex{float}			& \M
& Declares a decoration.\\

\envm{pull}		& \no				& \rexo{\inline}			& \no		& \no		& \ex{float}			& \M
& Declares a pull-quote.\\

\envm{equation}		& \no				& \rexo{\inline}			& \opt		& \dep		& \ex{float}			& \M
& Declares a wrapper around a \texttt{mathtex} or \texttt{mathml} block.\\

\envm{printout}		& \no				& \rexo{\inline}			& \opt		& \dep		& \ex{float}			& \M
& Declares a wrapper around a \texttt{source} block.\\

\envm{table}		& \no				& \rexo{\inline}			& \opt		& \dep		& \ex{float}			& \M
& Declares a wrapper around a \texttt{tabular} block.\\

\envm{figure}		& \no				& \rexo{\inline}			& \opt		& \dep		& \ex{float}			& \M
& Declares a wrapper around a \texttt{verbatim}, \texttt{image}, or \texttt{subpage} block.\\

\simm{part}		& \no				& \inline				& \depz		& \opt		& \no				& \M
& Declares a document part (the highest level division).\\

\simm{appendix}		& \no				& \no					& \no		& \opt		& \no				& \M
& Declares the beginning of the document's appendix.\\

\simm{section}		& \simm{h1}			& \inline				& \depz		& \opt		& \no				& \M
& Declares a new document section.\\

\simm{subsection}	& \simm{h2}			& \inline				& \depz		& \opt		& \no				& \M
& Declares a new document sub-section.\\

\simm{subsubsection}	& \simm{h3}			& \inline				& \depz		& \opt		& \no				& \M
& Declares a new document sub-sub-section.\\

\simm{bibliography}	& \no				& \no					& \no		& \opt		& \no				& \M
& Inserts all bibliography entries.\\

\simm{notes}		&\no				& \no					& \no		& \opt		& \no				& \M
& Inserts all document notes.\\

\simm{toc}		& \no				& \no					& \no		& \opt		& \no				& \M
& Inserts the document's table of contents.\\

\simm{title}		& \no				& \inline				& \no		& \no		& \no				& \M
& Declares the document title.\\

\simm{subtitle}		& \no				& \inline				& \no		& \no		& \no				& \M
& Declares a document subtitle.\\

\envm{abstract}		& \no				& \no					& \no		& \no		& \no				& \M
& Declares a block of text to be the document's abstract.\\

\simm{rule}		& \simm{hr}			& \no					& \no		& \no		& \no				& \M
& Inserts an horizontal separator.\\

\envm{bib}		& \no				& \no					& \no		& \no		& \no				& \M
& Declares a bibliography entry.\\

\envm{note}		& \no				& \no					& \no		& \no		& \no				& \M
& Declares a document note.\\

\envm{macrodef}		& \no				& \raw\raw\inline			& \no		& \no		& \no				& \M
& Macro definition.\\

\envm{boxoutdef}	& \no				& \raw\rexo{\inline\rexo{\raw}}		& \no		& \no		& \no				& \M
& Boxout definition.\\

\envm{theoremdef}	& \no				& \raw\inline\rexo{\raw}		& \no		& \no		& \no				& \M
& Theorem definition.\\

\bottomrule

\end{tabular}



\begin{tabular}{rlllllcl}

\toprule

			&				& \multicolumn{4}{c}{\hd{Parameters}}						& & \\

\cmidrule{3-6}\\

\hd{Command}		& \hd{Synonyms}			& \hd{Primary}			& \hd{Order}	& \hd{Label}	& \hd{Extra}	& \hd{T}
& \hd{Description}\\

\midrule

\simc{item}		& \simc{li}			& \no				& \no		& \no		& \no		& \C
& Declares a new item in an ordered or unordered list.\\

\simc{item}		& \simc{li}			& \inline			& \no		& \no		& \no		& \C
& Declares a new item in a description list.\\

\simc{question}		& \no				& \rexo{\inline}		& \no		& \no		& \no		& \C
& Declares a new custom question in a \texttt{qanda} block.\\

\simc{rquestion}	& \no				& \no				& \no		& \no		& \no		& \C
& Declares a new repeated question in a \texttt{qanda} block.\\

\simc{answer}		& \no				& \rexo{\inline}		& \no		& \no		& \no		& \C
& Declares a new custom answer in a \texttt{qanda} block.\\

\simc{ranswer}		& \no				& \no				& \no		& \no		& \no		& \C
& Declares a new repeated answer in a \texttt{qanda} block.\\

\simc{head}		& \no				& \no				& \no		& \no		& \no		& \C
& Declares the start of the head rows in a tabular environment.\\

\simc{body}		& \no				& \no				& \no		& \no		& \no		& \C
& Declares the start of a block of body rows in a tabular environment.\\

\simc{foot}		& \no				& \no				& \no		& \no		& \no		& \C
& Declares the start of the foot rows in a tabular environment.\\

\simm{who}		& \no				& \inline			& \no		& \no		& \no		& \M
& Declares the author in a bibliography entry.\\

\simm{what}		& \no				& \inline			& \no		& \no		& \no		& \M
& Declares the title in a bibliography entry.\\

\simm{where}		& \no				& \inline			& \no		& \no		& \no		& \M
& Declares the resource where the bibliography entry may be located.\\

\simm{\emph{macro}}	& \no				& \rexp{\inline}		& \no		& \no		& \no		& \M
& Invokes a macro.\\

\simm{arg}		& \no				& \raw				& \no		& \no		& \no		& \M
& Argument reference in macro definition.\\

\envm{\emph{boxout}}	& \no				& \rexo{\inline}		& \no		& \no		& \ex{float}	& \M
& Declares a boxout.\\

\envm{\emph{theorem}}	& \no				& \rexo{\inline}		& \no		& \no		& \ex{float}	& \M
& Declares a theorem.\\

\bottomrule

\end{tabular}

\end{document}

